\documentclass{article}
\usepackage{graphicx}
\usepackage[style=ieee]{biblatex} % Establecer el estilo de las referencias como IEEE
\usepackage{xcolor}
\usepackage{hyperref}
\usepackage{titletoc}
\usepackage{adjustbox}
\usepackage[spanish]{babel}

\hypersetup{
    colorlinks=true,
    linkcolor=blue, % Color del texto del enlace
    urlcolor=blue % Color del enlace
}

\usepackage{longtable} % Agrega el paquete longtable

\definecolor{mygreen}{RGB}{0,128,0}

\usepackage{array} % Para personalizar la tabla
\usepackage{booktabs} % Para líneas horizontales de mejor calidad
\usepackage{graphicx} % Paquete para incluir imágenes
\usepackage{float}
\usepackage[section]{placeins}

% Definir márgenes
\usepackage[margin=1in]{geometry}

\renewcommand{\contentsname}{\textcolor{mygreen}{Tabla de Contenidos}}

\begin{document}

\begin{titlepage}
    \centering
    % Logo de la Universidad
    \includegraphics[width=0.48\textwidth]{logo_universidad.png}
    \par\vspace{2cm}

    % Nombre de la Universidad y detalles del curso
    {\Large \textbf{Universidad Nacional de Colombia} \par}
    \vspace{0.5cm}
    {\large Ingeniería de Sistemas y Computación \par}
    {\large 2025969 Modelos estocásticos y simulación en computación y comunicaciones (01)\par}
    \vspace{3cm}

    % Detalles del laboratorio y actividad
    {\large \textbf{Tarea 31} \par}
    {\large Más sobre el dilema del prisionero\par}
    \vspace{3cm}

    % Lista de integrantes
    {\large \textbf{Integrantes:} \par}
    \vspace{0.5cm}
    \begin{tabular}{ll}
    Javier Andrés Tarazona Jiménez & jtarazonaj@unal.edu.co \\
    Yenifer Yulieth Mora Segura & ymoras@unal.edu.co \\
    Jefferson Duvan Ramirez Castañeda & jeramirezca@unal.edu.co \\
    \end{tabular}
    \par\vspace{3cm}

    % Fecha
    {\large Junio 26 de 2025 \par}
\end{titlepage}

\tableofcontents % Inserta la tabla de contenidos

\newpage % Salto de página para separar la tabla de contenidos del contenido del documento

% Contenido del artículo----------------------------------------------------------

%---------------------------------------------------------------------------------
% Intro --------------------------------------------------------------------------
%---------------------------------------------------------------------------------

\section{Introducción}\label{sec:intr}

%---------------------------------------------------------------------------------
% Marco Teórico ------------------------------------------------------------------
%---------------------------------------------------------------------------------

\section{Marco Teórico}\label{sec:marc}


\subsection{1. La Teoría de Juegos}
La teoría de juegos es la disciplina matemática que estudia las decisiones estratégicas de 
agentes interdependientes. Cada agente busca maximizar su propio beneficio, pero sus resultados 
dependen de las acciones de los demás. Los conceptos fundamentales incluyen:

\begin{itemize}
  \item \textbf{Jugadores}: agentes que toman decisiones.
  \item \textbf{Estrategias}: conjunto de acciones posibles para cada jugador.
  \item \textbf{Pagos (Payoffs)}: recompensas o pérdidas asociadas a cada combinación de 
    estrategias.
  \item \textbf{Equilibrio de Nash}: situación en la que ningún jugador puede mejorar 
    su pago cambiando unilateralmente su propia estrategia.
  \item \textbf{Estrategia Dominante}: Para cada jugador, no cooperar (traicionar) es mejor 
    sin importar la decisión del otro.
\end{itemize}

\subsection{2. Definición del Dilema del Prisionero}
El Dilema del Prisionero es un juego no cooperativo de dos jugadores, con las siguientes
características:

\begin{enumerate}
  \item \textbf{Hipótesis básica}: dos individuos aislados no pueden comunicarse.
  \item \textbf{Opciones de estrategia}: cooperar (\emph{guardar silencio}) o 
    traicionar (\emph{confesar}) - no cooperar.
  \item \textbf{Matriz de pagos}:\newline
  \begin{itemize}
    \item Ambos cooperan: pena moderada para cada uno.
    \item Uno coopera, otro traiciona: el traidor sale libre, el cooperador recibe la pena máxima.
    \item Ambos traicionan: pena intermedia para cada uno.
  \end{itemize}
\end{enumerate}

Matemáticamente, si denotamos la recompensa por mutua cooperación como $R$, 
la tentación de traicionar como $T$, la sanción por traición mutua como $P$ y la poca recompensa 
por ser traicionado como $S$, se cumple la desigualdad:

\[
T > R > P > S.
\]

\subsection{3. Estrategias y Equilibrio}
\begin{itemize}
  \item \textbf{Estrategia dominante}: para cada jugador, "traicionar" maximiza el pago 
    independiente de la decisión del otro.
  \item \textbf{Equilibrio de Nash}: $(traicionar / traicionar)$, aunque el resultado 
    colectivo óptimo sería la cooperación mutua.
\end{itemize}

\subsection{4. Extensiones y Repetición de Juegos}
Para modelar dinámicas más realistas, se considera el juego repetido:

\begin{itemize}
  \item \textbf{Repetición finita}: descompone la cooperación escalonada.
  \item \textbf{Repetición indeterminada}: permite estrategias condicionadas 
    (por ejemplo, \emph{Tit for Tat}), que fomentan la cooperación sostenida.
\end{itemize}

\subsection{5. Aplicación en Modelos Estocásticos y Simulación}
En sistemas de comunicaciones, el Dilema del Prisionero se emplea para:

\begin{itemize}
  \item Analizar protocolos de cooperación entre nodos (por ejemplo, reenvío de paquetes).
  \item Evaluar mecanismos de castigo y reputación.
  \item Explorar políticas de enrutamiento colaborativo bajo incertidumbre.
\end{itemize}

La simulación estocástica permite cuantificar cómo varían las tasas de cooperación 
según parámetros de castigo, beneficio y horizonte temporal.

%---------------------------------------------------------------------------------
% Descripción y Justificación del Problema a Resolver ----------------------------
%---------------------------------------------------------------------------------

\section{Descripción y Justificación del Problema a Resolver}\label{sec:descr}

En esta tarea se profundiza en el Dilema del Prisionero desde una perspectiva complementaria a 
la presentada en clase. Para ello, con base al episodio titulado “Lo que el Dilema del 
Prisionero Revela Sobre la Vida, el Universo y Todo lo Demás”, del canal Veritasium, en el 
que el Dr. Muller ilustra con ejemplos prácticos cómo distintas estrategias interactúan en 
un juego iterado de cooperación versus traición [\ref{ref:vidIntro}].

En principio se debe:
\begin{itemize}
  \item Revisar y analizar el contenido audiovisual propuesto, comprendiendo los fundamentos 
    del Dilema del Prisionero y la dinámica de las estrategias exhibidas 
    (especialmente la estrategia “Tit for Tat”).

  \item Diseñar y codificar dos nuevas estrategias propias

  \item Una benévola, que tienda a favorecer la cooperación inicial y perdonar 
    ocasionalmente la traición del oponente.

  \item Una malévola, que aproveche al máximo las oportunidades para traicionar al adversario, 
    buscando maximizar su propio beneficio.

  \item Implementar un simulador estocástico, donde cada par de estrategias 
    (benévola vs. Tit for Tat, malévola vs. Tit for Tat, y benévola vs. malévola) 
      se enfrente en múltiples rondas sucesivas bajo el esquema clásico de penalizaciones y 
      recompensas del Dilema del Prisionero.

  \item Estimar la eficacia de cada estrategia midiendo, a través de estadísticas de simulación, 
    el promedio de "pago" obtenido (nivel de cooperación lograda o traiciones realizadas) 
      en condiciones repetidas.

  \item El desafío consiste en traducir esos principios teóricos y audiovisuales a un 
  experimento computacional para definir cual es el enfoque mas eficiente.
\end{itemize}


El Dilema del Prisionero no solo es un pilar de la teoría de juegos, sino que en la práctica 
constituye un modelo esencial para simular el comportamiento de agentes en entornos estocásticos. 
Sus aplicaciones abarcan desde redes de comunicación distribuidas y protocolos de reenvío de 
paquetes hasta procesos evolutivos, dinámicas humanas de conflicto y competición en múltiples 
ámbitos. Entenderlo a fondo y experimentar con estrategias diversas es clave para diseñar 
sistemas robustos y predecir comportamientos en escenarios complejos.

En concreto, El Dilema sirve como un paradigma clave en simulaciones multiagente para modelar 
decisiones de cooperación y conflicto en entornos estocásticos, donde agentes con 
comportamientos pavlovianos y estrategias diversas interactúan en redes espaciales complejas 
para reproducir dinámicas reales de comunicación y competencia .
En sistemas distribuidos, esta lógica inspira protocolos de reenvío de paquetes y mecanismos de 
reputación que refuerzan o castigan conductas según el historial de cooperación.
Por su parte, en biología evolutiva, el juego explica la emergencia de comportamientos altruistas 
y de enjambre bajo presión selectiva.
Del mismo modo, la teoría se extiende a dinámicas humanas y bélicas, como el “security dilemma”, 
donde medidas defensivas generan espirales de desconfianza estatal, como la guerra fría. Pero
en ámbitos competitivos como el deporte y la economía, retrata por qué las estrategias 
egoístas conducen a resultados colectivos subóptimos. Finalmente, 
la experimentación mediante simulaciones iteradas permite calibrar parámetros de castigo, 
beneficio y perdón desde enfoques benignos hasta agresivos para diseñar protocolos robustos 
en sistemas ad hoc y distribuidos, fusionando teoría de juegos con práctica 
computacional avanzada [\ref{ref:just}].

%---------------------------------------------------------------------------------
% Diseño de la solución ---------------------------------------------------------
%---------------------------------------------------------------------------------

\section{Diseño de la solución}\label{sec:dis}

% Diseño de Solución para el simulador de agentes en el Dilema del Prisionero
\section*{Diseño de Solución}

El sistema se organiza en torno a una jerarquía de clases que modelan agentes capaces de jugar iteraciones del 
Dilema del Prisionero, junto con un módulo de simulación ("torneo") que empareja y hace competir a estos agentes. 
La solución consta de los siguientes componentes:

\begin{enumerate}
  \item \textbf{Estructura de Clases}
    \begin{description}
      \item[Clase base \texttt{Agente}] Define el comportamiento común a todos los jugadores:
        \begin{itemize}
          \item \emph{Atributos:}
            \begin{itemize}
              \item \texttt{state (int)}: última acción tomada (COOP o NOT\_COOP).
              \item \texttt{rounds\_played (int)}: número de rondas ya jugadas.
              \item \texttt{prev\_opponent\_response (Optional[int])}: respuesta anterior del oponente.
              \item \texttt{opponent\_consec\_count (int)}: conteo de respuestas idénticas consecutivas del oponente.
            \end{itemize}
          \item \emph{Métodos:}
            \begin{itemize}
              \item \texttt{start() -> int}: inicializa los contadores y devuelve la acción de la primera ronda.
              \item \texttt{choose(prev\_opponent\_response: int) -> int}: actualiza el estado interno, valida la respuesta del rival y elige la siguiente acción.
              \item \texttt{\_validate\_response(...)}: comprueba que la señal recibida sea COOP o NOT\_COOP.
              \item \texttt{\_choose\_coop()} y \texttt{\_choose\_not\_coop()}: manejan la lógica de cambio de \texttt{state} cuando el agente está en modo cooperar o no cooperar.
            \end{itemize}
        \end{itemize}
      \item[Subclases concretas]
        \begin{description}
          \item[\texttt{ElChance} (benévola):]
            \begin{itemize}
              \item Primera ronda: coopera siempre.
              \item Cuando coopera: sigue cooperando con probabilidad 95\%, 5\% de aleatoriedad. Si el oponente no coopera dos rondas seguidas, cambia a NO\_COOP.
              \item Cuando no coopera: permanece en NO\_COOP hasta que el oponente coopere, pero con probabilidad 10\% decide perdonar y cooperar.
            \end{itemize}
          \item[\texttt{Convenceme} (malévola):]
            \begin{itemize}
              \item Primera ronda: NO\_COOP.
              \item Cuando no coopera: mantiene NO\_COOP hasta que el oponente coopere, salvo una probabilidad 5\% de "piedad" donde coopera.
              \item Cuando coopera: lo hace al menos dos rondas, si el oponente vuelve a traicionar, regresa a NO\_COOP, tras cualquier par de cooperaciones, puede castigar con 10 rondas de NO\_COOP.
            \end{itemize}
          \item[\texttt{TitForTat}:]
            \begin{itemize}
              \item Primera ronda: COOP.
              \item Si el oponente cooperó, coopera, si traicionó, traiciona.
            \end{itemize}
        \end{description}
    \end{description}

  \item \textbf{Módulo de Simulación ("Torneo")}\\
    \texttt{torneo(agentes: List[Agente], n\_rondas: int) -> Resultados}
    \begin{enumerate}
      \item \emph{Inicialización:} para cada agente se invoca \texttt{start()} y se almacena la acción inicial.
      \item \emph{Iteración de rondas (2 a n\_rondas):}
        \begin{itemize}
          \item Para cada par ordenado $(A,B)$:
            \begin{enumerate}
              \item $a = A.choose(prev\_B)$ usando la última respuesta de $B$.
              \item $b = B.choose(prev\_A)$ usando la última respuesta de $A$.
              \item Registrar pagos según la matriz y actualizar historiales.
            \end{enumerate}
        \end{itemize}
      \item \emph{Cálculo de métricas:} al finalizar, para cada agente:
        \begin{itemize}
          \item Puntuación total (suma de pagos).
          \item Tasa de cooperación (n\_COOP / n\_rondas).
          \item Victorias relativas (veces con mayor pago).
        \end{itemize}
    \end{enumerate}

  \item \textbf{Flujo de Ejecución}
    \begin{enumerate}
      \item Definir instancias de \texttt{ElChance}, \texttt{Convenceme} y \texttt{TitForTat}.
      \item Configurar torneo: número de rondas y tipo de emparejamientos.
      \item Ejecutar \texttt{torneo(...)} y recolectar resultados.
      \item Analizar resultados (tablas o gráficos con \texttt{matplotlib} o \texttt{pandas}).
    \end{enumerate}

\end{enumerate}

Este diseño modular facilita la incorporación de nuevas estrategias y 
la extensión de la simulación a variantes del Dilema del Prisionero.


%---------------------------------------------------------------------------------
% Código Fuente ---------------------------------------------------------
%---------------------------------------------------------------------------------

\section{Código Fuente}\label{sec:cod}

El código fuente completo de este modelo se encuentra adjunto en el buzón 
(31 Tarazona Jimenez Javier Andres 02.zip)
y disponible en el repositorio GitHub del proyecto:

\begin{center}
\url{URL}
\end{center}

El repositorio contiene:
\begin{itemize}
\item A
\item B
\item C
\end{itemize}

%---------------------------------------------------------------------------------
% Manual Usuario ---------------------------------------------------------
%---------------------------------------------------------------------------------

\section{Manual Usuario}\label{sec:man_u}

El primer paso es descargar el archivo \texttt{11 Tarazona Jimenez Javier Andres 02.zip}.

Una vez descargado, descomprímalo y acceda a la carpeta. Dentro de ella, cree un 
entorno virtual utilizando Python 3.12. o superior. Para ello, ejecute el siguiente 
comando en 
la terminal o línea de comandos:

\begin{itemize}
  \item En Windows:
  \begin{verbatim}
    python3.12 -m venv nombre_del_entorno
  \end{verbatim}
  \item En macOS o Linux:
  \begin{verbatim}
    python3.12 -m venv nombre_del_entorno
  \end{verbatim}
\end{itemize}

Donde \texttt{nombre\_del\_entorno} es el nombre que desea asignar a su entorno virtual. 
A continuación, active el entorno virtual:

\begin{itemize}
  \item En Windows:
  \begin{verbatim}
    .\nombre_del_entorno\Scripts\activate
  \end{verbatim}
  \item En macOS o Linux:
  \begin{verbatim}
    source nombre_del_entorno/bin/activate
  \end{verbatim}
\end{itemize}

En el archivo \texttt{constants/program.py} encontrará las constantes del programa. 
En ese archivo, podrá modificar los parámetros de entrada que se detallan más abajo.\\

Después de configurar los parámetros, asegúrese de tener el entorno virtual activado. 
Una vez activo, puede ejecutar el archivo principal con el siguiente comando:

\begin{center}
  \begin{adjustbox}{minipage=\linewidth, center}
  \begin{verbatim}
    python main.py
  \end{verbatim}
  \end{adjustbox}
\end{center}

%---------------------------------------------------------------------------------
% Manual Técnico ---------------------------------------------------------
%---------------------------------------------------------------------------------

\section{Manual Técnico}\label{sec:man_t}


\subsection{Fases de la Simulación}


\subsection{Manejo de Datos}

\subsection{Evaluación de la Simulación}


\subsection{Conclusiones y Recomendaciones}


%---------------------------------------------------------------------------------
% Experimentación ---------------------------------------------------------
%---------------------------------------------------------------------------------

\section{Experimentación}\label{sec:exp}

\subsection{Análisis de resultados}

\subsubsection{Escenario 1: }

\subsubsection{Escenario 2: }
 
\subsubsection{Escenario 3: }



\section{Referencias}
\renewcommand{\refname}{}

\begin{thebibliography}{9}

\bibitem{ref} \label{ref:vidIntro} Veritasium en español, “Lo que el Dilema del Prisionero Revela 
Sobre la Vida, el Universo y Todo lo Demás,” YouTube video, 7:03, Mar. 2024. [Online]. 
Available: \url{https://youtu.be/vBgrvVY1jGo}

\bibitem{ref} \label{ref:just} R. Axelrod, The Evolution of Cooperation. New York: 
Basic Books, 1984.

\end{thebibliography}

\end{document}