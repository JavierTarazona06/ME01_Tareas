\documentclass{article}
\usepackage{graphicx}
\usepackage[style=ieee]{biblatex} % Establecer el estilo de las referencias como IEEE
\usepackage{xcolor}
\usepackage{hyperref}
\usepackage{titletoc}

\hypersetup{
    colorlinks=true,
    linkcolor=blue, % Color del texto del enlace
    urlcolor=blue % Color del enlace
}

\usepackage{longtable} % Agrega el paquete longtable

\definecolor{mygreen}{RGB}{0,128,0}

\usepackage{array} % Para personalizar la tabla
\usepackage{booktabs} % Para líneas horizontales de mejor calidad
\usepackage{graphicx} % Paquete para incluir imágenes
\usepackage{float}

% Definir márgenes
\usepackage[margin=1in]{geometry}

\renewcommand{\contentsname}{\textcolor{mygreen}{Tabla de Contenidos}}

\begin{document}

\begin{titlepage}
    \centering
    % Logo de la Universidad
    \includegraphics[width=0.48\textwidth]{logo_universidad.png}
    \par\vspace{2cm}

    % Nombre de la Universidad y detalles del curso
    {\Large \textbf{Universidad Nacional de Colombia} \par}
    \vspace{0.5cm}
    {\large Ingeniería de Sistemas y Computación \par}
    {\large 2025969 Modelos estocásticos y simulación en computación y comunicaciones (01)\par}
    \vspace{3cm}

    % Detalles del laboratorio y actividad
    {\large \textbf{Tarea 6} \par}
    {\large El problema de la programación académica en la UNAL \par}
    \vspace{3cm}

    % Lista de integrantes
    {\large \textbf{Integrantes:} \par}
    \vspace{0.5cm}
    \begin{tabular}{ll}
    Javier Andrés Tarazona Jiménez & jtarazonaj@unal.edu.co \\
    Yenifer & @unal.edu.co \\
    Jefferson & @unal.edu.co \\
    \end{tabular}
    \par\vspace{3cm}

    % Fecha
    {\large Abril 15 de 2024 \par}
\end{titlepage}

\tableofcontents % Inserta la tabla de contenidos

\newpage % Salto de página para separar la tabla de contenidos del contenido del documento

% Contenido del artículo----------------------------------------------------------

%---------------------------------------------------------------------------------
% Intro --------------------------------------------------------------------------
%---------------------------------------------------------------------------------

\section{Introducción}\label{sec:intr}
Aquí si empieza el contenido del artículo de la sección \ref{sec:intr} y también hay una referencia [\ref{ref:bienUniv}].

%---------------------------------------------------------------------------------
% Marco Teórico ------------------------------------------------------------------
%---------------------------------------------------------------------------------

\section{Marco Teórico}\label{sec:marc}
Aquí empieza el contenido del artículo de la sección \ref{sec:intr} y también hay una referencia [\ref{ref:bienUniv}].

\subsection{Ejemplo Documentación}

\subsubsection{Contextualización de la solución}

Problema identificado: ¿Qué necesidad o dificultad buscaba resolver la universidad? (por ejemplo: congestión en cursos, planificación ineficiente, asignación no equitativa, etc.)

Entorno institucional: ¿Qué tipo de universidad es? ¿Cuántos estudiantes? ¿Qué tan complejo es su sistema académico?

Objetivos del sistema de asignación: (eficiencia, justicia, reducción de tiempos administrativos, etc.)

\subsubsection{Descripción general de la solución}
Nombre o enfoque de la solución: (por ejemplo: algoritmo de colonia de hormigas, programación entera, RSD, etc.)

Tipo de herramienta: ¿Es una herramienta automatizada, un modelo matemático, un sistema web, un algoritmo implementado?

Actores involucrados: (¿Está orientado a estudiantes, administrativos, docentes?)

\subsubsection{Naturaleza de la solución}
Aquí se refiere a la categoría técnica o metodológica de la solución:

¿Es un modelo de optimización matemática, una heurística, un sistema probabilístico, una metaheurística, un sistema de recomendación?

También puede implicar si es una solución centralizada (administración asigna) o descentralizada (los usuarios eligen con base en preferencias).

\subsubsection{Funcionamiento del sistema}
¿Cómo se realiza la asignación o planificación?

¿Qué datos se necesitan?

¿Cómo se expresan las preferencias?

¿Qué lógica sigue el algoritmo (en términos generales)?

¿Qué tan automático es?

¿Cómo se gestionan los conflictos o empates?

\subsubsection{Resultados o impacto}
¿Qué mejoras reportaron?

¿Se adoptó oficialmente?

¿Se continúa utilizando?

\subsubsection{Referencia}
Nombre del artículo o proyecto, universidad, y si es posible, enlace o fuente bibliográfica.

\subsection{Bunded Probabilistic Serial (BPS): Yenifer}
\subsection{Random Serial Dictatorship (RSD): Jefferson}

\subsection{Probabilistic Serial Rule (PBS)}

\subsubsection{Descripción general de la solución}

Propuesto por Bogomolnaia y Moulin (2001), el algoritmo Probabilistic Serial Rule, también llamado 
Probabilistic Serial Mechanism, es un 
mecanismo de asignación aleatoria para repartir bienes indivisibles (como cupos en asignaturas) 
respetando las preferencias de los participantes bajo un enfoque equitativo.

Algunos entornos donde el mismo se puede aplicar se encuentran en la asignación de cursos a estudiantes,
como en el caso de la Universidad de París y en la Universidad de Tsinghua, reparto de turnos o 
recursos indivisibles, y asignación de cupos de movilidad internacional o servicios públicos.

Se podría decir entonces que se usa cuando un grupo de personas desea obtener uno (o varios) 
recursos limitados y los mismos no se pueden dividir, es decir, son discretos. Y se quiere una 
asignación que sea justa y eficiente en promedio, sin usar dinero ni intercambios, es decir, una 
solución equitativa que sea directa.

\subsubsection{Naturaleza de la solución}

Este algoritmo tiene una intuición muy práctica. Suponga un proceso en el cual se encuentran 
participantes de un buffet. En el mismo, cada uno empieza a comer de forma simultánea y de 
forma continua el plato que desee comer, es decir, su preferido. Una vez que dicho plato termina,
los integrantes pasan a comer su segundo plato preferido y así sucesivamente hasta que se terminan 
los platos.\\

Esta solución cuenta con unas ventajas y otras desventajas.\\

\textbf{Ventajas}

\begin{itemize}
  \item El método es eficiente, ya que maximiza el bienestar de los integrantes según preferencias.
        Ellos mismos buscan el "orden".
  \item Es justo y equitativo, ya que ningún integrante tiene prioridades.
  \item No es un método manipulable por los integrantes como el RSD, ya que, en principio, hacer una 
        estrategia es una mala idea por parte de los participantes, pues no hay prioridad. Suponga que
        la mejor opción es seleccionar lo que más quieren, porque de lo contrario se pueden quedar sin 
        dicho recurso una vez se realice la repartición simultánea (Weak Strategy-Proofness).
\end{itemize}

\textbf{Desventajas}

\begin{itemize}
  \item La asignación que se realiza es netamente probabilística y no determinista, pues depende 
        de las elecciones aleatorias de los integrantes y sus preferencias subjetivas. Usualmente, el
        método viene con una segunda etapa de ronda de sorteos (\textit{lottery decomposition}).
  \item Su carácter de paralelismo y varias variables hace que sea más difícil de implementar que
        RSD o BPS.
\end{itemize}

\subsubsection{Funcionamiento del sistema}

De forma generalizada, tenemos $n$ integrantes de un evento con $m$ recursos para escoger, y cada uno con 
una disponibilidad de $d_{m}$. Cada integrante va a consumir, de forma paralela, un elemento de $m$, 
reduciendo su
disponibilidad asociada $d_{m}$ bajo una velocidad uniforme, es decir, todos a la misma velocidad.
Cuando $d_{m}$ es cero, los integrantes consumen su siguiente recurso $m$ preferido. Y así se continúa
hasta que todos los $d_{m}$ sean iguales a 0.

Como lo puede notar, este proceso no determina una asignación exacta, sino que es una asignación
probabilística. Otra forma de verlo es como una lotería justa, porque cada integrante tiene una 
probabilidad de obtener el recurso según sus preferencias y su comportamiento.

Al final del proceso, hay una interpretación probabilística y una fase de sorteo. En este punto, cada
integrante tiene una fracción de cada recurso $m$; este valor acumulado es la probabilidad de 
cada integrante de obtener el recurso en una asignación final. Aquí hay dos casos:\\

\begin{itemize}
  \item \textbf{Caso 1: Los integrantes pueden recibir varios recursos}\\
        Se asigna cada recurso de forma individual según la probabilidad registrada.
  \item \textbf{Caso 2: Cada integrante debe recibir exactamente un recurso}\\
        Hay que descomponer la matriz de probabilidades en una combinación convexa de matrices
        de permutación (Ej: Algoritmo de Birkhoff) para generar una asignación determinista a partir de las probabilidades.
\end{itemize}

\textbf{Consideraciones}

\begin{itemize}
  \item A pesar de que se señaló que la velocidad debe ser uniforme, se pueden plantear prioridades para
        ciertos integrantes aumentando su velocidad.
  \item Se asume que los agentes son neutralmente riesgosos. Es decir, que su satisfacción se mide 
        por el valor esperado de la utilidad obtenida de la asignación final. Es decir, que el mecanismo
        se basa en las preferencias que expresan los agentes en vez de su actitud frente al riesgo.
  \item Hay una resistencia a la manipulación si un agente conoce las preferencias de los otros
        y encuentra satisfacción en el riesgo. Pero este beneficio está acotado a lo sumo al 50\% de
        ganancia adicional, es decir, un factor de $\frac{3}{2}$.
  \item El método presenta equilibrio de Nash; hay perfiles de preferencias en los que ningún agente 
        puede mejorar unilateralmente su resultado.
  \item Bajo comparación lexicográfica decreciente, se puede calcular una respuesta óptima en tiempo 
        razonable a la manipulación estratégica. Pero cuando hay que maximizar la utilidad esperada
        para cada agente (si hay más de dos agentes), el problema se vuelve NP-Hard. Esto hace al 
        método más robusto contra manipulaciones.
\end{itemize}


\subsubsection{Referencia}
Vea [\ref{ref:PBS}].


%---------------------------------------------------------------------------------
% Descripción y Justificación del Problema a Resolver ----------------------------
%---------------------------------------------------------------------------------

\section{Descripción y Justificación del Problema a Resolver}\label{sec:descr}

La Universidad Nacional de Colombia, como uno de sus fines misionales, tiene por objetivo impartir 
conocimiento al país, en particular a su comunidad de estudiantes mediante la docencia. Para 
realizarlo, se ofrecen cursos (asignaturas) como actividades académicas; sin embargo, dicha oferta 
requiere recursos y la sincronización de horarios, docentes y aulas. Y al mismo tiempo, dicho proceso 
se ve entorpecido por la falta de recursos, la pésima administración, las garantías académicas que se 
usan con poca conciencia por integrantes de la comunidad estudiantil, como la libre inscripción y 
cancelación, o el aprovechamiento del sistema mediante estrategias de manipulación.

Aora bien, dicho problema se puede abordar mediante modelación estocástica, planteando una simulación 
que emplee algoritmos y las condiciones de la Universidad para encontrar una estrategia y solución 
óptima al problema real.

\subsection{Objetivo Principal}

\begin{itemize}
  \item Formular \textit{un modelo estocástico} que describa el problema de planificación académica 
        en la Universidad Nacional.
  \item \textit{Simular} ese modelo para encontrar al menos \textit{una solución óptima} 
        que contribuya a mitigar el caos descrito.
\end{itemize}

\textbf{Variables que afectan:}

\begin{itemize}
  \item Mala administración.
  \item Insuficiencia de recursos (humanos, físicos, logísticos).
  \item Comportamiento estratégico o irresponsable de estudiantes (inscripción y cancelación masiva).
\end{itemize}

%---------------------------------------------------------------------------------
% Diseño de la solución ---------------------------------------------------------
%---------------------------------------------------------------------------------

\section{Diseño de la solución}\label{sec:dis}

Para abordar la solución, se presentan las áreas y abstracciones que se van a abordar.

\begin{itemize}
  \item La oferta.
        \begin{itemize}
          \item Cursos.
          \item Docentes.
          \item Salones (y laboratorios).
          \item Horarios.
        \end{itemize}
  \item Demanda fluctuante y mal gestionada.
        \begin{itemize}
          \item Preferencias no explicitas.
          \item Manipulación.
          \item Cancelaciones y sobreinscripción.
        \end{itemize}
  \item Solución.Problemas
        \begin{itemize}
          \item Falta de coordinación.
          \item Falta de mecanismos eficientes de asignación.
        \end{itemize}
  \item \textbf{Objetivo esperado.}
        \begin{itemize}
          \item Obtener una planificación justa, óptima y automatizada.
        \end{itemize}
\end{itemize}

\subsection{Metodología}

Para el desarrollo se abordará el problema en fases. Y el programa se realizará a través de módulos
bajo una arquitectura de componentes.

\subsubsection{Preparación}

\textbf{Módulo de Oferta}

Este módulo tiene como objetivo generar una oferta académica simulada de cursos a partir de una base 
de datos inicial, seleccionando aleatoriamente docentes, aulas y horarios bajo una distribución 
uniforme. Esta etapa corresponde a una primera aproximación estocástica del sistema, sin aún aplicar 
criterios de optimización o restricciones complejas.\\

Constantes del programa (Se pueden tomar de una base de datos o emularlas):
\begin{itemize}
  \item \textbf{Cursos:} Lista de diccionarios, donde cada elemento representa un curso con los siguientes atributos:
    \begin{itemize}
      \item \texttt{codigo} (int): identificador único del curso.
      \item \texttt{nombre} (str): nombre del curso.
      \item \texttt{creditos} (int): número de créditos académicos del curso.
      \item \texttt{programado} (bool): si ya se programó el curso.
    \end{itemize}
    
  \item \textbf{Docentes:} Lista de diccionarios, cada uno con:
    \begin{itemize}
      \item \texttt{cedula} (int): identificador del docente.
      \item \texttt{nombre} (str): nombre completo del docente.
      \item \texttt{programado} (bool): si ya se agendó a un curso.
    \end{itemize}

  \item \textbf{Aulas:} Lista de diccionarios, cada uno con:
    \begin{itemize}
      \item \texttt{edificio} (int): número del edificio.
      \item \texttt{aula\_id} (str): identificador del aula.
      \item \texttt{tipo} (str): tipo de espacio (e.g., ``laboratorio'', ``salón'').
      \item \texttt{programado} (bool): si ya se agendó a un curso.
    \end{itemize}

  \item \textbf{Horarios disponibles:} Lista de diccionarios con:
    \begin{itemize}
      \item \texttt{dia} (str): día de la semana, de lunes a sábado.
      \item \texttt{franja} (str): intervalo de tiempo disponible 
            (e.g., 7-9, 9-11, 11-13, 14-16, 16-18, 18-20, 7-13, 14-18).
    \end{itemize}
\end{itemize}

Para cada curso, el módulo va a realizar lo siguiente:

\begin{enumerate}
  \item \textbf{Selección de docente:} Se elige aleatoriamente un docente disponible 
        (que no esté asignado a otro curso en el mismo horario) y se marca como programado.
  
  \item \textbf{Selección de aula:} Se selecciona aleatoriamente un aula disponible 
        (que no esté ocupada en el horario asignado) y se marca como ocupada.

  \item \textbf{Selección de horario:} Se selecciona aleatoriamente un horario. Para garantizar 
        una distribución balanceada de clases a lo largo de la semana, se propone el siguiente 
        esquema de asignación adaptativa:

  \begin{itemize}
    \item Llevar un conteo de asignaciones por cada día de la semana.
    \item Definir una función de probabilidad que asigne mayor peso a los días menos cargados.
    \item Aplicar una probabilidad ponderada inversamente a la carga actual de cada día:
      $$
      P(dia_{i}) \propto \frac{1}{1 + asignaciones:dia_{i}}
      $$
    \item Excluir los días viernes y sábado inicialmente. Estos se incluirán únicamente cuando 
          reste el 5\% de los cursos por asignar.
  \end{itemize}

  \item \textbf{Creación de la clase:} Se registra la asignación como un nuevo diccionario en 
        la colección \texttt{clases}, la cual contiene los siguientes campos: 
        \texttt{codigo\_clase} generado aleatoriamente, \texttt{codigo\_curso}, 
        \texttt{cedula\_docente}, 
        \texttt{horario} y \texttt{aula} (identificada como \texttt{edificio+aula\_id}).
        Además, se asigna el número de \texttt{cupos} como un valor aleatorio entre 30 y 60, 
        calibrado de modo que el promedio de cupos se acerque al valor proporcionado por el usuario.
\end{enumerate}



\textbf{Módulo de Estudiantes}

Aquí se definirán los deseos de los estudiantes.

\begin{itemize}
  \item \textbf{Estudiante:} Lista de diccionarios con
        \begin{itemize}
          \item Cédula estudiante (int)
          \item Nombre estudiante (str)
          \item Lista de deseos de tamaño n donde cada elemento es un código de clase.
          \item Lista de asignaciones, las asiganturas que tiene inscritas
        \end{itemize}
\end{itemize}

Para cada estudiante se crea una lista de deseos seleccionando aleatoriamente, bajo una distribución 
uniforme, un código de clase y agregándolo a dicha lista.

\subsubsection{Modelo Estocástico: Proceso de Asignación}

Ahora bien, una vez que la oferta de clases ya está construida y los estudiantes también están 
definidos junto con sus preferencias, sigue el proceso de asignación y selección de asignaturas. 
Para realizar dicho proceso, se plantean tres escenarios donde la asignación se realiza mediante 
los algoritmos Bundled Probabilistic Serial, Random Serial Dictatorship y Probabilistic Serial Rule.\\


\textbf{Probabilistic Serial Rule (PS)}

\begin{itemize}
  \item \textbf{Inicialización}
        \begin{itemize}
          \item Para cada estudiante, se crea una lista vacía de asignaciones (inicialmente fraccional).
          \item Cada estudiante dispone de una lista ordenada de preferencias de asignaturas.
          \item Para cada asignatura, se conoce la capacidad (número de cupos disponibles).
          \item Se inicializa una diccionario de consumo $C[e,a] = 0$, donde $e$ denota al estudiante y 
          $a$ a la asignatura.
        \end{itemize}

  \item \textbf{Consumo continuo}
        \begin{itemize}
          \item Se simula un avance en el tiempo continuo. Al inicio, cada estudiante comienza a 
          "consumir" su asignatura más preferida que aún disponga de cupos, a una tasa unitaria.
          \item Mientras la asignatura no se agote (es decir, mientras su capacidad sea mayor que cero), 
          el estudiante sigue consumiéndola.
          \item Cuando una asignatura se consume completamente (cupos = 0), todos los estudiantes que la 
          estaban consumiendo pasan a consumir su siguiente asignatura preferida con cupos disponibles.
          \item Durante este proceso se actualiza el diccionario de consumo, de modo que, para cada 
          asignatura $a$, la suma de los consumos de todos los estudiantes satisface:
          $$
          \sum_{e} C[e,a] = capacidadInicial: a.
          $$
        \end{itemize}

  \item \textbf{Finalización y redondeo}
        \begin{itemize}
          \item Al finalizar el proceso, cada entrada $C[e,a]$ representa la fracción de la 
          asignatura $a$ "consumida" por el estudiante $e$. Dichas fracciones se interpretan como 
          las probabilidades de asignación.
          \item Se aplicará una técnica de \textit{rounding} (por ejemplo, mediante el método 
          de Birkhoff von Neumann) para convertir la asignación fraccional en asignaciones definitivas, 
          asignando a cada estudiante una o más asignaturas según se requiera.
        \end{itemize}
\end{itemize}

\subsubsection{Métricas}

En este punto ya se tiene la preparación del entorno y la simulación de asignación de asignaturas, 
es decir, los estudiantes ya pudieron inscribir las materias que estuvieron a su alcance. Ahora 
queda por determinar si dicha solución fue óptima o no. Con este objetivo, se analizarán cuatro 
métricas: porcentaje de estudiantes que obtuvieron su primera, segunda y tercera opción; equidad 
(medida como la desviación estándar de la satisfacción); número de conflictos de horario; y 
eficiencia de Pareto.\\





%---------------------------------------------------------------------------------
% Código Fuente ---------------------------------------------------------
%---------------------------------------------------------------------------------

\section{Código Fuente}\label{sec:cod}
Aquí empieza el contenido del artículo de la sección \ref{sec:intr} y también hay una referencia [\ref{ref:bienUniv}].


%---------------------------------------------------------------------------------
% Manual Usuario ---------------------------------------------------------
%---------------------------------------------------------------------------------

\section{Manual Usuario}\label{sec:man_u}

\textbf{Datos de entrada}

\begin{itemize}
  \item Número de cursos
  \item Número de docentes, debe ser mayor o igual al número de cursos
  \item Número de aulas, debe ser mayor o igual al número de cursos
  \item Número de estudiantes
  \item Máximo número de materias deseadas de un estudiante (Predeterminado=9).
  \item Promedio de número de cupos de las clases (Predeterminado 40).
  \item Método de asignación (BPS, RSD, PBS).
    \item Si elige BPS, debe elegir, el número maximo de asignaturas por Bundle.
\end{itemize}


%---------------------------------------------------------------------------------
% Manual Técnico ---------------------------------------------------------
%---------------------------------------------------------------------------------

\section{Manual Técnico}\label{sec:man_t}
Aquí empieza el contenido del artículo de la sección \ref{sec:intr} y también hay una referencia [\ref{ref:bienUniv}].


%---------------------------------------------------------------------------------
% Experimentación ---------------------------------------------------------
%---------------------------------------------------------------------------------

\section{Experimentación}\label{sec:exp}
Aquí empieza el contenido del artículo de la sección \ref{sec:intr} y también hay una referencia [\ref{ref:bienUniv}].

\subsection{Análisis de resultados}

\subsubsection{Escenario 1}

\subsubsection{Escenario 2}

\subsubsection{Escenario 3}


\renewcommand{\tablename}{Tabla}
\begin{table}[htbp]
  \centering
  \caption{Ejemplo de tabla 4x3 con contorno marcado \label{tab:Prueba}}
  \begin{tabular}{|c|c|c|c|} % 4 columnas, todas centradas y con contorno marcado
    \hline
    Encabezado 1 & Encabezado 2 & Encabezado 3 & Encabezado 4 \\
    \hline
    Celda 1 & Celda 2 & Celda 3 & Celda 4 \\
    \hline
    Celda 5 & Celda 6 & Celda 7 & Celda 8 \\
    \hline
    Celda 9 & Celda 10 & Celda 11 & Celda 12 \\
    \hline
  \end{tabular}
\end{table}

\begin{table}[H]
\centering
\caption{Nivel de Agua en los Embalses}
\label{tab:NivelAgua}
\begin{tabular}{|p{0.333\linewidth}|p{0.333\linewidth}|p{0.333\linewidth}|}
\hline
\textbf{Fecha} & \textbf{Estado} & \textbf{Nivel de los embalses/Capacidad} \\
\hline
Abril 2024 & Inicio Racionamiento & 14\% \\
\hline
Junio 2024 & Condiciones Actuales Escasez & 30\% \\
\hline
Ideal & Capacidad Completa & 100\% \\
\hline
\end{tabular}
\end{table}

Así mismo hay una tabla \ref{tab:Prueba}.

Tabla larga y personalizable en \ref{tab:anaTab}

\renewcommand{\tablename}{Tabla}
\begin{longtable}{|p{3.5cm}|p{3cm}|p{3cm}|p{5cm}|} % 4 columnas, todas centradas y con contorno marcado
  \caption{Análisis Stakeholders\label{tab:anaTab}} \\
  \hline
  \textbf{Stakeholder} & \textbf{Necesidades} & \textbf{Intereses} & \textbf{Impacto} \\
  \hline
  \endfirsthead % Encabezado de la primera página
  \hline
  \textbf{Stakeholder} & \textbf{Necesidades} & \textbf{Intereses} & \textbf{Impacto} \\
  \hline
  \endhead % Encabezado de las siguientes páginas
  CD1 & CD2 & CD3 & CD4 \\
  \hline
  CD5 & CD6 & CD7 & CD8 \\
  \hline
\end{longtable}


Y hay una figura \ref{fig:ejemplo}.\\

\renewcommand{\figurename}{Figura}
\begin{figure}[htbp]
  \centering
  \includegraphics[width=0.3\textwidth]{logo_universidad.png}
  %\includegraphics[scale=0.2]{logo_universidad.png}
  \caption{Descripción de la figura.}
  \label{fig:ejemplo}
\end{figure}

Por último una lista:

\begin{itemize}
  \item Item 1
  \item Item 2
  \item Item 3\\\\
\end{itemize}


Use la siguiente página para documentar las referencias en formato IEEE: 
\href{https://www.citethisforme.com/}{https://www.citethisforme.com/}

\section{Referencias}
\renewcommand{\refname}{}
\begin{thebibliography}{9}

\bibitem{ref} \label{ref:bienUniv}D. N. de B. Universitario, “Dirección Nacional de Bienestar Universitario,” contador de visitas. [Online]. Available: \href{https://bienestar.unal.edu.co/index.php}{https://bienestar.unal.edu.co/index.php}. [Accessed: 05-Mar-2023].  

\bibitem{ref} \label{ref:PBS} A. Bogomolnaia and H. Moulin, “A new solution to the random 
assignment problem,” Journal of Economic Theory, vol. 100, no. 2, pp. 295–328, 2001, doi: 
10.1006/jeth.2000.2710.



\end{thebibliography}

\end{document}