\documentclass{article}
\usepackage{graphicx}
\usepackage[style=ieee]{biblatex} % Establecer el estilo de las referencias como IEEE
\usepackage{xcolor}
\usepackage{hyperref}
\usepackage{titletoc}

\hypersetup{
    colorlinks=true,
    linkcolor=blue, % Color del texto del enlace
    urlcolor=blue % Color del enlace
}

\usepackage{longtable} % Agrega el paquete longtable

\definecolor{mygreen}{RGB}{0,128,0}

\usepackage{array} % Para personalizar la tabla
\usepackage{booktabs} % Para líneas horizontales de mejor calidad
\usepackage{graphicx} % Paquete para incluir imágenes
\usepackage{float}

% Definir márgenes
\usepackage[margin=1in]{geometry}

\renewcommand{\contentsname}{\textcolor{mygreen}{Tabla de Contenidos}}

\begin{document}

\begin{titlepage}
    \centering
    % Logo de la Universidad
    \includegraphics[width=0.48\textwidth]{logo_universidad.png}
    \par\vspace{2cm}

    % Nombre de la Universidad y detalles del curso
    {\Large \textbf{Universidad Nacional de Colombia} \par}
    \vspace{0.5cm}
    {\large Ingeniería de Sistemas y Computación \par}
    {\large 2025969 Modelos estocásticos y simulación en computación y comunicaciones \par}
    \vspace{3cm}

    % Detalles del laboratorio y actividad
    {\large \textbf{Tarea 6} \par}
    {\large El problema de la programación académica en la UNAL \par}
    \vspace{3cm}

    % Lista de integrantes
    {\large \textbf{Integrantes:} \par}
    \vspace{0.5cm}
    \begin{tabular}{ll}
    Javier Andrés Tarazona Jiménez & jtarazonaj@unal.edu.co \\
    Yenifer & @unal.edu.co \\
    Jefferson & @unal.edu.co \\
    \end{tabular}
    \par\vspace{3cm}

    % Fecha
    {\large Abril 15 de 2024 \par}
\end{titlepage}

\tableofcontents % Inserta la tabla de contenidos

\newpage % Salto de página para separar la tabla de contenidos del contenido del documento

% Contenido del artículo----------------------------------------------------------

%---------------------------------------------------------------------------------
% Intro --------------------------------------------------------------------------
%---------------------------------------------------------------------------------

\section{Introducción}\label{sec:ejemplo}
Aquí empieza el contenido del artículo de la sección \ref{sec:ejemplo} y también hay una referencia [\ref{ref:bienUniv}].

%---------------------------------------------------------------------------------
% Marco Teórico ------------------------------------------------------------------
%---------------------------------------------------------------------------------

\section{Marco Teórico}\label{sec:ejemplo}
Aquí empieza el contenido del artículo de la sección \ref{sec:ejemplo} y también hay una referencia [\ref{ref:bienUniv}].

%---------------------------------------------------------------------------------
% Descripción y Justificación del Problema a Resolver ----------------------------
%---------------------------------------------------------------------------------

\section{Descripción y Justificación del Problema a Resolver}\label{sec:ejemplo}
Aquí empieza el contenido del artículo de la sección \ref{sec:ejemplo} y también hay una referencia [\ref{ref:bienUniv}].

%---------------------------------------------------------------------------------
% Diseño de la solución ---------------------------------------------------------
%---------------------------------------------------------------------------------

\section{Diseño }\label{sec:ejemplo}
Aquí empieza el contenido del artículo de la sección \ref{sec:ejemplo} y también hay una referencia [\ref{ref:bienUniv}].



\renewcommand{\tablename}{Tabla}
\begin{table}[htbp]
  \centering
  \caption{Ejemplo de tabla 4x3 con contorno marcado \label{tab:Prueba}}
  \begin{tabular}{|c|c|c|c|} % 4 columnas, todas centradas y con contorno marcado
    \hline
    Encabezado 1 & Encabezado 2 & Encabezado 3 & Encabezado 4 \\
    \hline
    Celda 1 & Celda 2 & Celda 3 & Celda 4 \\
    \hline
    Celda 5 & Celda 6 & Celda 7 & Celda 8 \\
    \hline
    Celda 9 & Celda 10 & Celda 11 & Celda 12 \\
    \hline
  \end{tabular}
\end{table}

\begin{table}[H]
\centering
\caption{Nivel de Agua en los Embalses}
\label{tab:NivelAgua}
\begin{tabular}{|p{0.333\linewidth}|p{0.333\linewidth}|p{0.333\linewidth}|}
\hline
\textbf{Fecha} & \textbf{Estado} & \textbf{Nivel de los embalses/Capacidad} \\
\hline
Abril 2024 & Inicio Racionamiento & 14\% \\
\hline
Junio 2024 & Condiciones Actuales Escasez & 30\% \\
\hline
Ideal & Capacidad Completa & 100\% \\
\hline
\end{tabular}
\end{table}

Así mismo hay una tabla \ref{tab:Prueba}.

Tabla larga y personalizable en \ref{tab:anaTab}

\renewcommand{\tablename}{Tabla}
\begin{longtable}{|p{3.5cm}|p{3cm}|p{3cm}|p{5cm}|} % 4 columnas, todas centradas y con contorno marcado
  \caption{Análisis Stakeholders\label{tab:anaTab}} \\
  \hline
  \textbf{Stakeholder} & \textbf{Necesidades} & \textbf{Intereses} & \textbf{Impacto} \\
  \hline
  \endfirsthead % Encabezado de la primera página
  \hline
  \textbf{Stakeholder} & \textbf{Necesidades} & \textbf{Intereses} & \textbf{Impacto} \\
  \hline
  \endhead % Encabezado de las siguientes páginas
  CD1 & CD2 & CD3 & CD4 \\
  \hline
  CD5 & CD6 & CD7 & CD8 \\
  \hline
\end{longtable}


Y hay una figura \ref{fig:ejemplo}.\\

\renewcommand{\figurename}{Figura}
\begin{figure}[htbp]
  \centering
  \includegraphics[width=0.3\textwidth]{logo_universidad.png}
  %\includegraphics[scale=0.2]{logo_universidad.png}
  \caption{Descripción de la figura.}
  \label{fig:ejemplo}
\end{figure}

Por último una lista:

\begin{itemize}
  \item Item 1
  \item Item 2
  \item Item 3\\\\
\end{itemize}


Use la siguiente página para documentar las referencias en formato IEEE: 
\href{https://www.citethisforme.com/}{https://www.citethisforme.com/}

\section{Referencias}
\renewcommand{\refname}{}
\begin{thebibliography}{9}

\bibitem{ref} \label{ref:bienUniv}D. N. de B. Universitario, “Dirección Nacional de Bienestar Universitario,” contador de visitas. [Online]. Available: \href{https://bienestar.unal.edu.co/index.php}{https://bienestar.unal.edu.co/index.php}. [Accessed: 05-Mar-2023].  



\end{thebibliography}

\end{document}