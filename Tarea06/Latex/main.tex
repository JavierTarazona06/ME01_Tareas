\documentclass{article}
\usepackage{graphicx}
\usepackage[style=ieee]{biblatex} % Establecer el estilo de las referencias como IEEE
\usepackage{xcolor}
\usepackage{hyperref}
\usepackage{titletoc}

\hypersetup{
    colorlinks=true,
    linkcolor=blue, % Color del texto del enlace
    urlcolor=blue % Color del enlace
}

\usepackage{longtable} % Agrega el paquete longtable

\definecolor{mygreen}{RGB}{0,128,0}

\usepackage{array} % Para personalizar la tabla
\usepackage{booktabs} % Para líneas horizontales de mejor calidad
\usepackage{graphicx} % Paquete para incluir imágenes
\usepackage{float}

% Definir márgenes
\usepackage[margin=1in]{geometry}

\renewcommand{\contentsname}{\textcolor{mygreen}{Tabla de Contenidos}}

\begin{document}

\begin{titlepage}
    \centering
    % Logo de la Universidad
    \includegraphics[width=0.48\textwidth]{logo_universidad.png}
    \par\vspace{2cm}

    % Nombre de la Universidad y detalles del curso
    {\Large \textbf{Universidad Nacional de Colombia} \par}
    \vspace{0.5cm}
    {\large Ingeniería de Sistemas y Computación \par}
    {\large 2025969 Modelos estocásticos y simulación en computación y comunicaciones (01)\par}
    \vspace{3cm}

    % Detalles del laboratorio y actividad
    {\large \textbf{Tarea 6} \par}
    {\large El problema de la programación académica en la UNAL \par}
    \vspace{3cm}

    % Lista de integrantes
    {\large \textbf{Integrantes:} \par}
    \vspace{0.5cm}
    \begin{tabular}{ll}
    Javier Andrés Tarazona Jiménez & jtarazonaj@unal.edu.co \\
    Jefferson Duvan Ramirez Castañeda & jeramirezca@unal.edu.co \\
    Yenifer Yulieth Mora Segura & ymoras@unal.edu.co \\
    \end{tabular}
    \par\vspace{3cm}

    % Fecha
    {\large Abril 15 de 2024 \par}
\end{titlepage}

\tableofcontents % Inserta la tabla de contenidos

\newpage % Salto de página para separar la tabla de contenidos del contenido del documento

% Contenido del artículo----------------------------------------------------------

%---------------------------------------------------------------------------------
% Intro --------------------------------------------------------------------------
%---------------------------------------------------------------------------------

\section{Introducción}\label{sec:intr}
La Universidad Nacional enfrenta diversos problemas administrativos que afectan la programación 
académica, entre ellos, la asignación de horarios, la distribución de cupos y la gestión de recursos.
 Estos problemas se ven agravados por prácticas como la inscripción y cancelación indiscriminada de 
 asignaturas, lo que impide a muchos estudiantes acceder a los cursos que necesitan.

Este trabajo tiene como objetivo investigar y analizar soluciones basadas en algoritmos utilizados
 en universidades prestigiosas, como el Bunded Probabilistic Serial (BPS), el Random Serial Dictatorship 
 (RSD) y el Probabilistic Serial Rule (PSR), para mejorar la asignación de cursos. 
 Además, se explorarán otras soluciones implementadas en diferentes instituciones. 
 A partir de esta investigación, se desarrollará un modelo estocástico que describa la situación y 
 permita encontrar soluciones óptimas mediante simulación. La propuesta busca optimizar el proceso 
 de asignación académica, promoviendo la equidad y eficiencia en la Universidad Nacional.

%---------------------------------------------------------------------------------
% Marco Teórico ------------------------------------------------------------------
%---------------------------------------------------------------------------------

\section{Marco Teórico}\label{sec:marc}

\subsection{Contextualización del problema}

La asignación de horarios de cursos en instituciones educativas es un desafío complejo, especialmente cuando
 los estudiantes tienen preferencias sobre combinaciones de cursos que se ajusten a sus necesidades y 
 restricciones personales. Los métodos tradicionales, como el orden de llegada (First-Come, First-Served - FCFS),
  suelen resultar en asignaciones ineficientes y generan envidia entre los estudiantes debido a la falta de 
  equidad en el proceso. Para abordar estas limitaciones, se han propuesto diversos algoritmos que se describirán en las secciones presentadas a continuación.

\subsection{Bundled Probabilistic Serial (BPS)}

\subsubsection{Descripción general de la solución}

El BPS es un mecanismo de asignación aleatoria que permite a los estudiantes expresar sus preferencias sobre combinaciones de cursos (paquetes) en lugar de cursos individuales. Este enfoque reconoce que los estudiantes valoran ciertos conjuntos de cursos que se ajustan a sus horarios y objetivos académicos. El BPS asigna probabilidades a cada estudiante para recibir diferentes paquetes de cursos, basándose en sus preferencias declaradas. Posteriormente, se realiza una asignación determinista mediante una lotería que respeta estas probabilidades, garantizando una distribución equitativa y eficiente de los cursos disponibles.

\subsubsection{Naturaleza de la solución}

El BPS es un mecanismo de naturaleza estocástica, ya que asigna probabilidades a los estudiantes para recibir ciertos paquetes de cursos. Sin embargo, la implementación final es determinista, ya que se utiliza una lotería para realizar la asignación concreta de cursos a cada estudiante. Este enfoque combina la equidad y eficiencia de los mecanismos aleatorios con la necesidad de asignaciones deterministas en la práctica. Además, el BPS satisface propiedades deseables como la eficiencia ordinal, la equidad (ausencia de envidia) y una forma débil de resistencia a la manipulación estratégica.

\subsubsection{Funcionamiento del sistema}

El algoritmo \textit{Bundled Probabilistic Serial} (BPS) está diseñado para abordar el problema de asignación de paquetes de cursos, donde cada paquete representa una combinación de recursos indivisibles (por ejemplo, horarios de cursos, laboratorios o seminarios). A continuación, se describe detalladamente el proceso de asignación mediante BPS:

\begin{itemize}
    \item \textbf{Elicitación de preferencias:} Los estudiantes deben expresar sus preferencias sobre los paquetes de cursos. Este proceso puede ser complejo, ya que el número de paquetes posibles crece exponencialmente con la cantidad de cursos disponibles.

    Para simplificar la tarea, el sistema proporciona una interfaz que permite clasificar paquetes en función de atributos clave (como horarios compatibles, cursos obligatorios y electivos preferidos). Además, se pueden emplear técnicas de reducción dimensional, como la agrupación de cursos en categorías o la limitación del número de opciones permitidas.

    \item \textbf{Consumo simultáneo de paquetes:} El mecanismo se basa en un enfoque similar al proceso de ``simultaneous eating'' del algoritmo \textit{Probabilistic Serial Rule} (PS).

    Durante un intervalo continuo de tiempo, todos los estudiantes comienzan a ``consumir'' su paquete de cursos preferido al mismo ritmo. Cuando un paquete alcanza su capacidad (por ejemplo, si uno de sus cursos se llena), el estudiante detiene su consumo de dicho paquete y pasa al siguiente en su lista de preferencias. Este proceso se repite hasta que se agotan todas las capacidades disponibles en la oferta académica.

    \item \textbf{Asignación de probabilidades:} Al finalizar el proceso de consumo, cada estudiante acumula una fracción de probabilidad asociada a cada paquete consumido. Estas probabilidades representan el tiempo relativo dedicado a consumir cada paquete antes de que dejara de estar disponible.

    \item \textbf{Descomposición en asignaciones deterministas:} Una vez calculadas las probabilidades, el sistema convierte estas fracciones en una asignación determinista a través de un proceso de descomposición de matriz probabilística.

    \begin{itemize}
        \item Cada fila de la matriz representa a un estudiante, y cada columna a un paquete de cursos.
        \item Los valores en la matriz indican la probabilidad acumulada de que un estudiante reciba un paquete específico.
        \item Para transformar esta matriz en asignaciones concretas, se utiliza el algoritmo de Birkhoff–von Neumann, que permite descomponer la matriz en una combinación convexa de matrices de permutación.
    \end{itemize}

    \item \textbf{Lotería para asignación final:} Con base en la descomposición anterior, se realiza una lotería que asigna a cada estudiante un paquete de forma determinista, respetando las probabilidades acumuladas. Este procedimiento garantiza una asignación factible, acorde con las capacidades y preferencias declaradas por los estudiantes.

    \item \textbf{Resultados del proceso}

    Al concluir el proceso de asignación, se observan los siguientes resultados:
    
    \begin{itemize}
        \item Cada estudiante recibe una asignación determinista que refleja sus preferencias y cumple con las restricciones de capacidad.
        
        \item Los paquetes asignados optimizan el uso de los recursos académicos, minimizando cursos no utilizados y asignaciones ineficientes.
    \end{itemize}

  \end{itemize}

\subsubsection{Propiedades adicionales y consideraciones}

\begin{itemize}
    \item \textbf{Velocidad uniforme:} En el proceso de consumo simultáneo, todos los estudiantes consumen a la misma velocidad. No obstante, esta puede ajustarse si se incorporan factores de prioridad (por ejemplo, año académico, urgencia de graduación, etc.).

    \item \textbf{Eficiencia y justicia:} BPS garantiza eficiencia ordinal, es decir, no se puede mejorar la asignación de un estudiante sin perjudicar a otro. Asimismo, promueve una distribución justa basada en preferencias declaradas.

    \item \textbf{Resistencia a la manipulación:} Aunque BPS no es completamente estrategia-proof, los incentivos para manipular preferencias son bajos. Las estrategias subóptimas tienden a resultar en asignaciones menos favorables, especialmente en comparación con mecanismos como RSD.
\end{itemize}

La implementación del BPS en la Universidad Técnica de Múnich ha demostrado mejoras significativas en comparación con métodos tradicionales como FCFS. Se observaron asignaciones más equitativas, una mayor satisfacción de los estudiantes y una reducción en la envidia entre los participantes. Además, el BPS permitió una mejor utilización de la capacidad de los cursos y facilitó la gestión de las preferencias complejas de los estudiantes sobre combinaciones de cursos.

\subsubsection{Referencia}

Vease [\ref{ref:BPS}] [\ref{ref:MatchingBundle}].

\subsection{Random Serial Dictatorship (RSD)}

\subsubsection{Descripción general de la solución}

El algoritmo \textit{Random Serial Dictatorship} (RSD), también conocido como 
\textit{Random Priority}, ha sido implementado en prestigiosas 
instituciones como Harvard, la Universidad de Boston y diversas 
universidades públicas de Turquía. Se trata de un mecanismo de asignación 
aleatoria que busca repartir recursos indivisibles como cupos en asignaturasde 
forma justa, aunque no necesariamente eficiente en diferentes sentidos.

En este mecanismo, los participantes son ordenados aleatoriamente, y cada uno, en su 
turno, elige su opción preferida entre las disponibles. Esta simplicidad lo hace 
especialmente útil en contextos donde la transparencia y la facilidad de implementación 
son prioritarias. En entornos universitarios, como el caso de la Universidad Nacional, 
RSD podría ser utilizado para asignar cupos de materias, turnos de laboratorios o 
incluso recursos como intercambios académicos.

\subsubsection{Naturaleza de la solución}

El funcionamiento de RSD se puede entender como una fila de personas que toman decisiones 
por turnos. Imaginemos que hay varios estudiantes haciendo fila para inscribir materias. 
Un estudiante es seleccionado aleatoriamente, y este escoge su materia preferida entre 
las que aún tienen cupo. Luego se selecciona al siguiente estudiante y así sucesivamente 
hasta que no queden más cupos.

Este algoritmo, si bien puede parecer muy simple, tiene una base teórica sólida en teoría de juegos y economía de asignación. Su principal atractivo radica en su resistencia a la manipulación estratégica y en su sencillez tanto conceptual como computacional.

\textbf{Ventajas}
\begin{itemize}
    \item \textbf{Simplicidad y transparencia}: Fácil de explicar, implementar y verificar, tanto para la administración como para los estudiantes.
    \item \textbf{Estrategia óptima}: Para cada estudiante, lo mejor que puede hacer es revelar sus verdaderas preferencias, lo cual se conoce como \textit{estrategia dominante}.
    \item \textbf{Misma probabilidad}: Dado que el orden se determina aleatoriamente, todos los participantes tienen la misma probabilidad de obtener una posición favorable.
\end{itemize}

\textbf{Desventajas}
\begin{itemize}
    \item \textbf{Ineficiencia ex-post}: Puede dar lugar a asignaciones que no son óptimas en términos del bienestar total. Por ejemplo, puede haber una reasignación alternativa en la que todos estén mejor sin violar las restricciones.
    \item \textbf{Sensibilidad al azar}: Los resultados pueden variar significativamente entre ejecuciones, lo cual puede percibirse como injusto desde una perspectiva individual.
    \item \textbf{No equitativo en preferencias}: Aunque el mecanismo es justo en cuanto al orden, no garantiza resultados equitativos respecto a la satisfacción de preferencias.
\end{itemize}

\subsubsection{Funcionamiento del sistema}

Formalmente, el mecanismo opera sobre un conjunto de $n$ agentes (por ejemplo, estudiantes) y $m$ recursos indivisibles (como cupos en materias). Se genera una permutación aleatoria de los agentes, $\pi = (\pi_1, \pi_2, \ldots, \pi_n)$, la cual define el orden de elección.

\begin{enumerate}
    \item Se inicializan los recursos con su disponibilidad total.
    \item El agente $\pi_1$ selecciona su recurso más preferido entre los disponibles.
    \item Se actualiza la disponibilidad y el turno pasa al agente $\pi_2$.
    \item El proceso continúa hasta que todos los agentes han elegido o los recursos se han agotado.
\end{enumerate}

Este procedimiento produce una asignación determinista una vez fijado el orden, pero el mecanismo completo es aleatorio en el sentido de que el orden de los agentes varía en cada ejecución.

\subsubsection{Consideraciones}

\begin{itemize}
    \item \textbf{Dominio restringido}: Aunque el método funciona en entornos de recursos indivisibles, no se adapta bien cuando hay complementariedad entre recursos (por ejemplo, dos materias que deben tomarse juntas).
    \item \textbf{No enmascara inequidades estructurales}: Si bien todos los estudiantes tienen la misma probabilidad de ser seleccionados primero, aquellos con preferencias más comunes pueden quedar sistemáticamente desfavorecidos.
    \item \textbf{Factible de implementar}: A diferencia de algoritmos más sofisticados como BPS o PBS, este mecanismo puede implementarse con facilidad en los sistemas actuales de la Universidad, sirviendo incluso como una solución de transición.
\end{itemize}

\subsubsection{Referencia}

% Bogomolnaia, A., \& Moulin, H. (2001). A new solution to the random assignment problem. \textit{Journal of Economic Theory}, 100(2), 295–328.

% Abdulkadiroğlu, A., \& Sönmez, T. (1998). Random serial dictatorship and the core from random endowments in house allocation problems. \textit{Econometrica}, 66(3), 689–701.

Vease [\ref{ref:RSD1}],[\ref{ref:RSD2}].

\subsection{Probabilistic Serial Rule (PBS)}

\subsubsection{Descripción general de la solución}

Propuesto por Bogomolnaia y Moulin (2001), el algoritmo Probabilistic Serial Rule, también llamado 
Probabilistic Serial Mechanism, es un 
mecanismo de asignación aleatoria para repartir bienes indivisibles (como cupos en asignaturas) 
respetando las preferencias de los participantes bajo un enfoque equitativo.

Algunos entornos donde el mismo se puede aplicar se encuentran en la asignación de cursos a estudiantes,
como en el caso de la Universidad de París y en la Universidad de Tsinghua, reparto de turnos o 
recursos indivisibles, y asignación de cupos de movilidad internacional o servicios públicos.

Se podría decir entonces que se usa cuando un grupo de personas desea obtener uno (o varios) 
recursos limitados y los mismos no se pueden dividir, es decir, son discretos. Y se quiere una 
asignación que sea justa y eficiente en promedio, sin usar dinero ni intercambios, es decir, una 
solución equitativa que sea directa.

\subsubsection{Naturaleza de la solución}

Este algoritmo tiene una intuición muy práctica. Suponga un proceso en el cual se encuentran 
participantes de un buffet. En el mismo, cada uno empieza a comer de forma simultánea y de 
forma continua el plato que desee comer, es decir, su preferido. Una vez que dicho plato termina,
los integrantes pasan a comer su segundo plato preferido y así sucesivamente hasta que se terminan 
los platos.\\

Esta solución cuenta con unas ventajas y otras desventajas.\\

\textbf{Ventajas}

\begin{itemize}
  \item El método es eficiente, ya que maximiza el bienestar de los integrantes según preferencias.
  Además no tiene en cuenta otros aspectos como las notas de semestres anteriores para determinar la
  asignación de citas.
        Ellos mismos buscan el "orden".
  \item Es justo y equitativo, ya que ningún integrante tiene prioridades.
  \item No es un método manipulable por los integrantes como el RSD, ya que, en principio, hacer una 
        estrategia es una mala idea por parte de los participantes, pues no hay prioridad. Suponga que
        la mejor opción es seleccionar lo que más quieren, porque de lo contrario se pueden quedar sin 
        dicho recurso una vez se realice la repartición simultánea (Weak Strategy-Proofness).
\end{itemize}

\textbf{Desventajas}

\begin{itemize}
  \item La asignación que se realiza es netamente probabilística y no determinista, pues depende 
        de las elecciones aleatorias de los integrantes y sus preferencias subjetivas. Usualmente, el
        método viene con una segunda etapa de ronda de sorteos (\textit{lottery decomposition}).
  \item Su carácter de paralelismo y varias variables hace que sea más difícil de implementar que
        RSD o BPS.
\end{itemize}

\subsubsection{Funcionamiento del sistema}

De forma generalizada, tenemos $n$ integrantes de un evento con $m$ recursos para escoger, y cada uno con 
una disponibilidad de $d_{m}$. Cada integrante va a consumir, de forma paralela, un elemento de $m$, 
reduciendo su
disponibilidad asociada $d_{m}$ bajo una velocidad uniforme, es decir, todos a la misma velocidad.
Cuando $d_{m}$ es cero, los integrantes consumen su siguiente recurso $m$ preferido. Y así se continúa
hasta que todos los $d_{m}$ sean iguales a 0.

Como lo puede notar, este proceso no determina una asignación exacta, sino que es una asignación
probabilística. Otra forma de verlo es como una lotería justa, porque cada integrante tiene una 
probabilidad de obtener el recurso según sus preferencias y su comportamiento.

Al final del proceso, hay una interpretación probabilística y una fase de sorteo. En este punto, cada
integrante tiene una fracción de cada recurso $m$; este valor acumulado es la probabilidad de 
cada integrante de obtener el recurso en una asignación final. Aquí hay dos casos:\\

\begin{itemize}
  \item \textbf{Caso 1: Los integrantes pueden recibir varios recursos}\\
        Se asigna cada recurso de forma individual según la probabilidad registrada.
  \item \textbf{Caso 2: Cada integrante debe recibir exactamente un recurso}\\
        Hay que descomponer la matriz de probabilidades en una combinación convexa de matrices
        de permutación (Ej: Algoritmo de Birkhoff) para generar una asignación determinista a partir de las probabilidades.
\end{itemize}

\textbf{Consideraciones}

\begin{itemize}
  \item A pesar de que se señaló que la velocidad debe ser uniforme, se pueden plantear prioridades para
        ciertos integrantes aumentando su velocidad.
  \item Se asume que los agentes son neutralmente riesgosos. Es decir, que su satisfacción se mide 
        por el valor esperado de la utilidad obtenida de la asignación final. Es decir, que el mecanismo
        se basa en las preferencias que expresan los agentes en vez de su actitud frente al riesgo.
  \item Hay una resistencia a la manipulación si un agente conoce las preferencias de los otros
        y encuentra satisfacción en el riesgo. Pero este beneficio está acotado a lo sumo al 50\% de
        ganancia adicional, es decir, un factor de $\frac{3}{2}$.
  \item El método presenta equilibrio de Nash; hay perfiles de preferencias en los que ningún agente 
        puede mejorar unilateralmente su resultado.
  \item Bajo comparación lexicográfica decreciente, se puede calcular una respuesta óptima en tiempo 
        razonable a la manipulación estratégica. Pero cuando hay que maximizar la utilidad esperada
        para cada agente (si hay más de dos agentes), el problema se vuelve NP-Hard. Esto hace al 
        método más robusto contra manipulaciones.
\end{itemize}


\subsubsection{Referencia}
Vea [\ref{ref:PBS}].

\subsection{Asignación de Cursos Universitarios mediante Mercados Competitivos}

\subsubsection{Contextualización de la solución}

En contextos universitarios, cada semestre se asignan horarios de cursos a millones de estudiantes. Este entorno se caracteriza por la diversidad de preferencias de los alumnos y por restricciones severas derivadas de la capacidad limitada de los cursos, los horarios, los requisitos académicos y otros factores, como la prioridad basada en la antigüedad o la pertenencia a determinados departamentos o programas.

Los métodos tradicionales, tales como el \textit{first-come-first-served} o los mecanismos simples de subasta, han demostrado ser ineficientes y pueden generar resultados poco equitativos, donde estudiantes con conexiones o ventajas estructurales obtienen todas las plazas deseadas, mientras que otros quedan con asignaciones subóptimas.

En respuesta a estos desafíos, Kornbluth y Kushnir proponen un novedoso mecanismo de asignación determinístico: el mecanismo de Pseudo-Mercado con Prioridades (PMP). Esta solución adopta la idea de equilibrio competitivo, pero adaptada a un entorno sin transferencias monetarias reales. El mecanismo emplea “dinero ficticio”, asignado casi equitativamente a los estudiantes, permitiendo traducir las preferencias cardinales declaradas en precios implícitos para cada curso.

Además, se incorpora explícitamente una estructura de prioridades para respetar elementos clave como la antigüedad o la pertenencia a ciertos grupos. Esto da lugar a un “corte” de prioridad: los estudiantes con mayor prioridad pueden acceder a ciertos cursos sin costo, mientras que quienes están por debajo deben “pagar” una fracción de su presupuesto ficticio. Así, el sistema busca minimizar el error de mercado, maximizar la utilidad global y garantizar un trato justo.

\subsubsection{Descripción General de la Solución}

El mecanismo PMP se compone de los siguientes pasos fundamentales:

\begin{itemize}
    \item \textbf{Recolección de preferencias:} Se solicita a cada estudiante que informe sus preferencias sobre los horarios y combinaciones de cursos aceptables (respetando restricciones como traslapes o prerrequisitos), utilizando un lenguaje de reporte diseñado para evitar sobrecarga cognitiva.
    
    \item \textbf{Asignación de presupuestos ficticios:} Cada alumno recibe un presupuesto dentro de un intervalo casi uniforme, por ejemplo $[1, 1 + \beta]$, donde $\beta$ es pequeño. Esto asegura equidad y refleja ligeras diferencias en la prioridad.
    
    \item \textbf{Cálculo del equilibrio competitivo aproximado:} Con base en las preferencias y presupuestos, se calcula una asignación de cursos junto con un vector de precios por curso y nivel de prioridad. Se utiliza una parametrización que reduce la dimensión efectiva del problema y facilita la búsqueda de un equilibrio donde el error de mercado no supere un valor teórico $\alpha$.
    
    \item \textbf{Optimización en dos fases:}
    \begin{itemize}
        \item \textit{Fase I:} Se ajustan los precios mediante un algoritmo inspirado en el proceso de tâtonnement (ajuste gradual de precios) para equilibrar oferta y demanda, minimizando el error de mercado.
        
        \item \textit{Fase II:} Una vez alcanzado el umbral teórico, se refina la asignación ajustando localmente los precios de los cursos sobreasignados para asegurar que no se exceda el límite de $k - 1$ asignaciones adicionales (siendo $k$ el número máximo de cursos que puede tomar un estudiante).
    \end{itemize}
\end{itemize}

\subsubsection{Naturaleza de la Solución}

La propuesta PMP es determinística y se fundamenta en conceptos de equilibrio competitivo, adaptados a un entorno sin transacciones monetarias reales. Sus características principales incluyen:

\begin{itemize}
    \item \textbf{Enfoque de pseudo-mercado:} El uso de dinero ficticio permite representar presupuestos casi iguales y calcular precios de equilibrio que respetan restricciones de capacidad y mejoran las asignaciones.
    
    \item \textbf{Integración de prioridades:} La estructura de prioridades (por ejemplo, por año académico o pertenencia a programas) influye directamente en los precios, permitiendo acceso preferente o gratuito a estudiantes con mayor prioridad.
    
    \item \textbf{Uso de preferencias cardinales:} A diferencia de mecanismos ordinales (como Gale–Shapley), PMP considera la intensidad de las preferencias, logrando asignaciones más eficientes y mayor satisfacción promedio.
    
    \item \textbf{Propiedades teóricas deseables:} El mecanismo garantiza estabilidad aproximada, eficiencia de Pareto y minimización de la envidia (limitada a un curso y entre estudiantes de similar prioridad).
\end{itemize}

\subsubsection{Funcionamiento del Sistema}

El funcionamiento del PMP se desarrolla en varias etapas:

\begin{enumerate}
    \item \textbf{Reporte de preferencias y presupuestos:}
    \begin{itemize}
        \item Los estudiantes registran sus preferencias sobre combinaciones válidas de cursos, indicando la utilidad asociada a cada opción.
        \item Se les asigna un presupuesto ficticio (e.g., entre $1$ y $1 + \beta$) que representa la cantidad de “créditos” disponibles para “pagar” por los cursos deseados.
    \end{itemize}

    \item \textbf{Cálculo de precios por prioridad:}

    Se define un vector de precios por curso y nivel de prioridad mediante la fórmula:

    \[
    p_{c,r}(t) = \max(t_c - (r - 1)b, 0)
    \]

    donde $t \in [0, Rb]^M$, $R$ es el número de niveles de prioridad y $b$ es el incremento de precio. Esta estructura genera un punto de corte (\textit{cutoff}) que determina qué estudiantes acceden sin costo y quiénes deben pagar, según su prioridad.

    \item \textbf{Fase I – Ajuste de precios y asignación inicial:}

    Se realiza un ajuste iterativo de precios vía tâtonnement, con el objetivo de igualar la demanda con la capacidad y minimizar el error de mercado.

    \item \textbf{Fase II – Refinamiento y control de sobreasignaciones:}

    Se ajustan los precios de cursos con sobreasignación para que no excedan el límite permitido de asignaciones adicionales ($\leq k - 1$). Si no se obtiene mejora, puede reiniciarse la fase con nuevos precios iniciales.
\end{enumerate}

\subsubsection{Resultados o Impacto}

Las simulaciones realizadas con datos reales de asignación de cursos demuestran que el mecanismo PMP ofrece mejoras significativas:

\begin{itemize}
    \item \textbf{Mayor utilidad promedio:} PMP logra la utilidad media estudiantil más alta en todas las categorías académicas, especialmente en primer y segundo año (mejoras del 9\% y 7\%, respectivamente).

    \item \textbf{Reducción de la desigualdad en la satisfacción:} Se observa una disminución considerable en la desviación estándar de las utilidades, con una mejora del 11.16\% en estudiantes de primer año, indicando mayor equidad.

    \item \textbf{Menor envidia:} Solo el 1.26\% de los estudiantes manifiesta envidia (deseo por el horario de otro estudiante con la misma o menor prioridad), frente a porcentajes superiores al 7\% en mecanismos como DA o RSD.

    \item \textbf{Control efectivo del error de mercado:} El error promedio se mantiene en torno a 14 asientos, muy por debajo del umbral teórico ($\alpha \approx 43.5$), demostrando la eficiencia del mecanismo.

    \item \textbf{Robustez y resistencia a manipulaciones:} En mercados grandes, PMP es prácticamente inmune a manipulaciones estratégicas, mostrando que es estrategia-proof en el largo plazo.
\end{itemize}

\subsubsection{Referencia}

Vease [\ref{ref:Mercados}].

\subsection{Asignación de Cursos Universitarios mediante Aprendizaje Automático}

\subsubsection{Contextualización de la solución}

La asignación de cursos en universidades a menudo enfrenta restricciones complejas y preferencias diversas por parte de los estudiantes. Tradicionalmente, este problema se ha abordado mediante métodos como optimización combinatoria o mecanismos económicos. Sin embargo, dichas soluciones no escalan bien con grandes poblaciones estudiantiles ni se adaptan con facilidad a contextos dinámicos.

Frente a estas limitaciones, se propone una solución basada en aprendizaje automático (*machine learning*) que aprende a asignar cursos de manera eficiente y equitativa, observando ejemplos anteriores de asignaciones exitosas y adaptándose a nuevas situaciones sin requerir una modelación manual detallada.

\subsubsection{Descripción general de la solución}

El sistema propuesto se basa en un enfoque de \textit{policy learning}, donde un modelo de aprendizaje supervisado o por refuerzo aprende una política de asignación de cursos. El proceso consta de las siguientes etapas:

\begin{itemize}
  \item \textbf{Representación del estado:} Cada estudiante y curso se representa mediante vectores de características que incluyen historial académico, preferencias declaradas, y restricciones del sistema.
  \item \textbf{Modelo de predicción:} Se entrena un modelo (por ejemplo, una red neuronal o un árbol de decisión) que predice la utilidad o conveniencia de asignar un curso a un estudiante dado el estado del sistema.
  \item \textbf{Optimización de la política:} A partir del modelo aprendido, se ejecuta una política que selecciona asignaciones maximizando una función de utilidad global (ej. satisfacción estudiantil o cobertura equitativa).
  \item \textbf{Entrenamiento iterativo:} El modelo se entrena de manera iterativa con datos simulados y reales, refinando sus decisiones conforme recibe retroalimentación del entorno.
\end{itemize}

\subsubsection{Naturaleza de la solución}

La solución se caracteriza por los siguientes aspectos clave:

\begin{itemize}
  \item \textbf{Aprendizaje basado en datos:} A diferencia de enfoques deterministas o basados en reglas fijas, esta propuesta aprende directamente de los datos históricos y de simulaciones.
  \item \textbf{Adaptabilidad:} El sistema puede ajustarse automáticamente ante cambios en la oferta académica, aumento en la población estudiantil o nuevas reglas institucionales.
  \item \textbf{Escalabilidad:} Al utilizar modelos eficientes de aprendizaje automático, la solución se escala a miles de estudiantes y cursos sin pérdida significativa de rendimiento.
  \item \textbf{Optimización multiobjetivo:} La política considera múltiples objetivos, como maximizar la utilidad total, reducir la desigualdad y evitar conflictos de horario.
\end{itemize}

\subsubsection{Funcionamiento del sistema}

El sistema funciona en ciclos semestrales y consta de los siguientes pasos operativos:

\begin{enumerate}
  \item \textbf{Recolección de datos:} Se recopilan preferencias estudiantiles, disponibilidad de cursos y restricciones del semestre.
  \item \textbf{Entrenamiento del modelo:} Se entrena o actualiza el modelo de asignación con base en datos anteriores y simulaciones.
  \item \textbf{Predicción y asignación:} El modelo predice la utilidad de asignar cada curso a cada estudiante y ejecuta la política de asignación.
  \item \textbf{Evaluación y retroalimentación:} Se mide el rendimiento (utilidad, equidad, eficiencia) y se usa la retroalimentación para mejorar el modelo en siguientes iteraciones.
\end{enumerate}

\subsubsection{Resultados o impacto}

Según los autores, la aplicación del modelo de aprendizaje automático ha mostrado resultados prometedores:

\begin{itemize}
  \item \textbf{Mayor utilidad global:} Se observa un incremento significativo en la satisfacción promedio estudiantil frente a métodos base.
  \item \textbf{Reducción en el conflicto de horarios:} La tasa de solapamientos de cursos asignados disminuyó considerablemente.
  \item \textbf{Menor desigualdad:} El sistema tiende a favorecer asignaciones más equitativas, evitando concentrar beneficios en estudiantes con ventajas previas.
  \item \textbf{Robustez ante cambios:} El modelo mantiene su eficacia incluso ante variaciones sustanciales en las preferencias o en la oferta académica.
\end{itemize}

\subsubsection{Referencia}

Vease [\ref{ref:AprendizajeAutomatico}].

% Kornbluth, D., & Kushnir, A. (2023). \textit{Machine Learning-powered Course Allocation}. Carnegie Mellon University / Harvard University.


%---------------------------------------------------------------------------------
% Descripción y Justificación del Problema a Resolver ----------------------------
%---------------------------------------------------------------------------------

\section{Descripción y Justificación del Problema a Resolver}\label{sec:descr}

La Universidad Nacional de Colombia, como uno de sus fines misionales, tiene por objetivo impartir 
conocimiento al país, en particular a su comunidad de estudiantes mediante la docencia. Para 
realizarlo, se ofrecen cursos (asignaturas) como actividades académicas; sin embargo, dicha oferta 
requiere recursos y la sincronización de horarios, docentes y aulas. Y al mismo tiempo, dicho proceso 
se ve entorpecido por la falta de recursos, la pésima administración, las garantías académicas que se 
usan con poca conciencia por integrantes de la comunidad estudiantil, como la libre inscripción y 
cancelación, o el aprovechamiento del sistema mediante estrategias de manipulación.

Aora bien, dicho problema se puede abordar mediante modelación estocástica, planteando una simulación 
que emplee algoritmos y las condiciones de la Universidad para encontrar una estrategia y solución 
óptima al problema real.

\subsection{Objetivo Principal}

\begin{itemize}
  \item Formular \textit{un modelo estocástico} que describa el problema de planificación académica 
        en la Universidad Nacional.
  \item \textit{Simular} ese modelo para encontrar al menos \textit{una solución óptima} 
        que contribuya a mitigar el caos descrito.
\end{itemize}

\textbf{Variables que afectan:}

\begin{itemize}
  \item Mala administración.
  \item Insuficiencia de recursos (humanos, físicos, logísticos).
  \item Comportamiento estratégico o irresponsable de estudiantes (inscripción y cancelación masiva).
\end{itemize}

%---------------------------------------------------------------------------------
% Diseño de la solución ---------------------------------------------------------
%---------------------------------------------------------------------------------

\section{Diseño de la solución}\label{sec:dis}

Para abordar la solución, se presentan las áreas y abstracciones que se van a abordar.

\begin{itemize}
  \item La oferta.
        \begin{itemize}
          \item Cursos.
          \item Docentes.
          \item Salones (y laboratorios).
          \item Horarios.
        \end{itemize}
  \item Demanda fluctuante y mal gestionada.
        \begin{itemize}
          \item Preferencias no explicitas.
          \item Manipulación.
          \item Cancelaciones y sobreinscripción.
        \end{itemize}
  \item Solución.Problemas
        \begin{itemize}
          \item Falta de coordinación.
          \item Falta de mecanismos eficientes de asignación.
        \end{itemize}
  \item \textbf{Objetivo esperado.}
        \begin{itemize}
          \item Obtener una asignación justa, óptima y automatizada.
        \end{itemize}
\end{itemize}

\subsection{Metodología}

Para el desarrollo se abordará el problema en fases. Y el programa se realizará a través de módulos
bajo una arquitectura de componentes.

\subsubsection{Preparación}

\textbf{Módulo de Oferta}

Este módulo tiene como objetivo generar una oferta académica simulada de cursos a partir de una base 
de datos inicial, seleccionando aleatoriamente docentes, aulas y horarios bajo una distribución 
uniforme. Esta etapa corresponde a una primera aproximación estocástica del sistema, sin aún aplicar
criterios de optimización o restricciones complejas.\\

Constantes del programa (Se pueden tomar de una base de datos o emularlas):
\begin{itemize}
  \item \textbf{Cursos:} Lista de diccionarios, donde cada elemento representa un curso con los siguientes atributos:
    \begin{itemize}
      \item \texttt{codigo} (str): identificador único del curso.
      \item \texttt{programado} (bool): si ya se programó el curso.
      \item \texttt{numGrupos} (int): numero de grupos que el curso puede tener.
      \item \texttt{clases\_asociadas} (list[str]): lista de codigos de las clases 
            relacionadas, que se crearán mas tarde
    \end{itemize}
    
  \item \textbf{Docentes:} Lista de diccionarios, cada uno con:
    \begin{itemize}
      \item \texttt{cedula} (str): identificador del docente.
      \item \texttt{programado} (bool): si ya se agendó a un curso.
    \end{itemize}

  \item \textbf{Aulas:} Lista de diccionarios, cada uno con:
    \begin{itemize}
      \item \texttt{edificio} (str): número del edificio.
      \item \texttt{aula\_id} (str): identificador del aula.
      \item \texttt{IDs} (edificio+aula\_id): identificador del aula.
      \item \texttt{programado} (bool): si ya se agendó a un curso.
    \end{itemize}

  \item \textbf{Horarios disponibles:} Lista de diccionarios con:
    \begin{itemize}
      \item \texttt{ID} (str)
      \item \texttt{dia} (list[int]): día(s) de la semana, de lunes a sábado.
      \item \texttt{franja} (str): intervalo de tiempo disponible 
            (e.g., 7-9, 9-11, 11-13, 14-16, 16-18, 18-20, 7-13, 14-18).
    \end{itemize}
\end{itemize}

Para cada curso (hasta que los grupos que puede tener sean cero), 
el módulo va a realizar lo siguiente:

\begin{enumerate}
  \item \textbf{Selección de docente:} Se elige aleatoriamente un docente disponible 
        (que no esté asignado a otro curso en el mismo horario) y se marca como programado.
  
  \item \textbf{Selección de aula:} Se selecciona aleatoriamente un aula disponible 
        (que no esté ocupada en el horario asignado) y se marca como ocupada.

  \item \textbf{Selección de horario:} Se selecciona aleatoriamente un horario. Para garantizar 
        una distribución balanceada de clases a lo largo de la semana, se propone el siguiente 
        esquema de asignación adaptativa:

  \begin{itemize}
    \item Llevar un conteo de asignaciones por cada día de la semana.
    \item Definir una función de probabilidad que asigne mayor peso a los días menos cargados.
    \item Aplicar una probabilidad ponderada inversamente a la carga actual de cada día:
      $$
      P(dia_{i}) \propto \frac{1}{1 + asignaciones:dia_{i}}
      $$
    \item Excluir los días viernes y sábado inicialmente. Estos se incluirán únicamente cuando 
          reste el 5\% de los cursos por asignar.
  \end{itemize}

  \item \textbf{Creación de la clase:} Se registra la asignación como un nuevo diccionario en 
        la colección \texttt{clases}, la cual contiene los siguientes campos: 
        \texttt{codigo\_clase} generado aleatoriamente, \texttt{codigo\_curso}, 
        \texttt{cedula\_docente}, 
        \texttt{horario} y \texttt{aula} (identificada como \texttt{edificio+aula\_id}).
        Además, se asigna el número de \texttt{cupos} como un valor aleatorio entre 30 y 60, 
        calibrado de modo que el promedio de cupos se acerque al valor proporcionado por el usuario.
        También se asigna el número de grupo \texttt{grupo} con base a la clase, y se resta uno 
        a la cantidad de
        grupos del curso.
        
        De la misma forma, se agrega el código de la clase a la lista de clases\_asociadas 
        del curso correspondiente\\

        Es así que la clase queda de la siguiente forma:\\

        \textbf{Clase:} Lista de diccionarios con:
        \begin{itemize}
          \item \texttt{codigo\_clase} (str)
          \item \texttt{codigo\_curso} (str): código del curso asociado
          \item \texttt{cedula\_docente} (str): código del docente asociado
          \item \texttt{horario} (Tuple[str, str]): horario asociado Dia: str, franja Horaria: str.
          \item \texttt{codigo\_aula} (str): aula asociada, edificio\+aula\_id
          \item \texttt{grupo} (int): número de grupo
          \item \texttt{cupos} (int): cupos asociados
        \end{itemize}
\end{enumerate}



\textbf{Módulo de Estudiantes}

Aquí se definirán los deseos de los estudiantes.

\begin{itemize}
  \item \textbf{Estudiante:} Lista de diccionarios con
        \begin{itemize}
          \item cedula (str): Cédula estudiante
          \item p.a.p.i (float): P.A.P.I estudiante 
          \item cantidad\_materias\_inscribir (int): 
                Numero de materias que va inscribir, menor o igual al tamaño
                de la lista de desos.
          \item cantidad\_materias\_deseadas (int): Tamaño de la lista de deseos
          \item lista\_preferencias (list[str]): Lista de deseos/preferencias de tamaño 
                n donde cada elemento es un 
                código de clase.
          \item lista\_materias\_asignadas (list[str]): Lista de asignaciones, las 
                asiganturas que tiene inscritas
        \end{itemize}
\end{itemize}

\textbf{*Nota:} El estudiante quiere inscribir 5 materias, pero puede tener 10 opcionadas.
10 va a ser el tanaño de la lista de preferencias.\\

Para cada estudiante se define un P.A.P.I aleatorio siguiendo una distribución
Gaussiano con valor medio 3.8 y una desviación estandar acorde.

Para cada estudiante, se define el número de materias que va a isncribir de forma aleatoria
entre 3 y el número del parámetro de entrada. Se crea una lista de deseos de un tamaño 
que se consigue tomando un Numero
aleatorio del parámetro de entrada y seleccionando aleatoriamente, bajo una distribución 
uniforme, un código de clase y agregándolo a dicha lista (entre 3 y el parámetro de entrada).

Ahora se deben asignar clases a la lista de deseos. Para esto en un ciclo for
desde 0 hasta $tamanioListaDeseos$. Tomar n curso aleatorio del pool, guardar sus clases en una 
lista de elegibles. Se toma una clase de esa lista aleatoriamente si presenta conflicto de horario 
con las ya seleccionadas, elimina la clase de elegibles y toma otra clase. Si definitivamente tiene 
conflictos con todas las clases de ese curso, descarta
esta elección. Independientemente de si selecciono o no una materia como deseada, el contador 
se reduce.

\subsubsection{Modelo Estocástico: Proceso de Asignación}

Ahora bien, una vez que la oferta de clases ya está construida y los estudiantes también están 
definidos junto con sus preferencias, sigue el proceso de asignación y selección de asignaturas. 
Se uará el algoritmo Probabilistic Serial Rule porque la teoría indicea que es Pareto-Eficiente.\\


\textbf{Probabilistic Serial Rule (PS)}

\begin{itemize}
  \item \textbf{Inicialización}
        \begin{itemize}
          \item Para cada estudiante, se crea una lista vacía de asignaciones (inicialmente fraccional).
          \item Cada estudiante dispone de una lista ordenada de preferencias de asignaturas.
          \item Para cada asignatura, se conoce la capacidad (número de cupos disponibles).
          \item Se inicializa una diccionario de consumo $C[e,a] = 0$, donde $e$ denota al estudiante y 
          $a$ a la asignatura.
        \end{itemize}

  \item \textbf{Consumo continuo}
        \begin{itemize}
          \item Se simula un avance en el tiempo continuo. Al inicio, cada estudiante comienza a 
          "consumir" su asignatura más preferida que aún disponga de cupos, a una tasa unitaria.
          \item Mientras la asignatura no se agote (es decir, mientras su capacidad sea mayor que cero), 
          el estudiante sigue consumiéndola.
          \item Cuando una asignatura se consume completamente (cupos = 0), todos los estudiantes que la 
          estaban consumiendo pasan a consumir su siguiente asignatura preferida con cupos disponibles.
          \item Durante este proceso se actualiza el diccionario de consumo, de modo que, para cada 
          asignatura $a$, la suma de los consumos de todos los estudiantes satisface:
          $$
          \sum_{e} C[e,a] = capacidadInicial: a.
          $$
        \end{itemize}

  \item \textbf{Finalización y redondeo}
        \begin{itemize}
          \item Al finalizar el proceso, cada entrada $C[e,a]$ representa la fracción de la 
          asignatura $a$ "consumida" por el estudiante $e$. Dichas fracciones se interpretan como 
          las probabilidades de asignación.
          \item Se aplicará una técnica de \textit{rounding} (Greedy Fraccional) para convertir 
          la asignación fraccional en asignaciones definitivas, 
          asignando a cada estudiante una o más asignaturas según se requiera.
          \item Para ese método se van a ordenar a los estudiantes por P.A.P.I.
        \end{itemize}
\end{itemize}

\section{Métricas de Evaluación}
Para cuantificar la calidad, equidad y eficiencia de las asignaciones en los cuatro escenarios, definimos las siguientes métricas, implementadas en \texttt{metrics.py}:

\subsection{Porcentaje de Opciones Obtenidas}
Sea $N$ el número total de estudiantes. Para cada estudiante $i$, denotamos por ${r_{i,1},\dots,r_{i,m_i}}$ las posiciones en su lista de preferencias de las materias que finalmente obtuvo. Entonces:
\begin{itemize}
\item Primera opción:
\(p_1 = \frac{1}{N} \sum_{i=1}^N \mathbf{1}\{\exists j:\;r_{i,j}=1\}\)
\item Segunda opción:
\(p_2 = \frac{1}{N} \sum_{i=1}^N \mathbf{1}\{\exists j:\;r_{i,j}=2\}\)
\item Tercera opción:
\(p_3 = \frac{1}{N} \sum_{i=1}^N \mathbf{1}\{\exists j:\;r_{i,j}=3\}\)
\item Todas tres primeras:
\(p_{123} = \frac{1}{N} \sum_{i=1}^N \mathbf{1}\{\{1,2,3\} \subseteq \{r_{i,j}\}\}\)
\end{itemize}

\subsection{Satisfacción Normalizada}
Para cada estudiante $i$ con $m_i$ asignaciones, calculamos la\emph{ satisfacción obtenida}:

$$
S_i = \sum_{j=1}^{m_i} \frac{1}{r_{i,j}},
$$

donde $r_{i,j}$ es la posición de la $j$-ésima asignatura asignada. El máximo posible, si recibiera siempre sus primeras \$m\_i\$ opciones, es:

$$
S_i^{\max} = \sum_{k=1}^{m_i} \frac{1}{k}.
$$

La satisfacción normalizada de $i$ es entonces:

$$
\widetilde S_i = \frac{S_i}{S_i^{\max}} \quad (0\le\widetilde S_i\le1).
$$

La\emph{ satisfacción media} global es:

$$
\overline S = \frac{1}{N} \sum_{i=1}^N \widetilde S_i.
$$

\subsection{Equidad (Desviación Estándar)}
La equidad se mide como la dispersión de las satisfacciones normalizadas:

Valores bajos ($\sigma_S \approx0$) indican distribuciones homogéneas de satisfacción; valores altos revelan disparidad.

$$
\sigma_S = \sqrt{ \frac{1}{N} \sum_{i=1}^N (\widetilde S_i - \overline S)^2 }.
$$

\subsection{Eficiencia de Pareto}
Un estudiante $i$ se dice \emph{no dominado} si no existe otro estudiante 
$j$ con $\widetilde{S_j} > \widetilde S_i$. Denotando por $U$ el
 conjunto de no dominados, la eficiencia de Pareto es:

$$
E_P = \frac{|U|}{N}.
$$

Un $E_P$ cercano a 1 implica que pocas asignaciones pueden mejorarse sin empeorar otras, reflejando eficiencia SD.

Estas cuatro métricas proporcionan un marco sólido para comparar políticas de asignación y evaluar su impacto en equidad, eficiencia y satisfacción estudiantil.


%---------------------------------------------------------------------------------
% Código Fuente ---------------------------------------------------------
%---------------------------------------------------------------------------------

\section{Código Fuente}\label{sec:cod}


%---------------------------------------------------------------------------------
% Manual Usuario ---------------------------------------------------------
%---------------------------------------------------------------------------------

\section{Manual Usuario}\label{sec:man_u}

El primer paso es descargar el archivo \texttt{Tarea06.zip}.

Una vez descargado, descomprímalo y acceda a la carpeta. Dentro de ella, cree un 
entorno virtual utilizando Python 3.12. Para ello, ejecute el siguiente comando en 
la terminal o línea de comandos:

\begin{itemize}
  \item En Windows:
  \begin{verbatim}
    python3.12 -m venv nombre_del_entorno
  \end{verbatim}
  \item En macOS o Linux:
  \begin{verbatim}
    python3.12 -m venv nombre_del_entorno
  \end{verbatim}
\end{itemize}

Donde \texttt{nombre\_del\_entorno} es el nombre que desea asignar a su entorno virtual. 
A continuación, active el entorno virtual:

\begin{itemize}
  \item En Windows:
  \begin{verbatim}
    .\nombre_del_entorno\Scripts\activate
  \end{verbatim}
  \item En macOS o Linux:
  \begin{verbatim}
    source nombre_del_entorno/bin/activate
  \end{verbatim}
\end{itemize}

En el archivo \texttt{constants/program.py} encontrará las constantes del programa. 
En ese archivo, podrá modificar los parámetros de entrada que se detallan más abajo.\\

Después de configurar los parámetros, ejecute el archivo \texttt{main.py} utilizando 
el entorno virtual.

\textbf{Datos de entrada}

\begin{itemize}
  \item Número de cursos.
  \item Número máximo de grupos por curso.
  \item Número mínimo de grupos por curso.
  \item Promedio de número de grupos por curso.
  \item Desviación estandar de número de grupos por curso.
  \item Número de docentes, debe ser mayor o igual al número de cursos por maximo de grupos.
  \item Número de aulas, debe ser mayor o igual al número de cursos por maximo de grupos.
  \item Número máximo de cupos por curso.
  \item Número mínimo de cupos por curso.
  \item Promedio de número de cupos por curso.
  \item Desviación estandar de número de cupos por curso.
  \item Número de estudiantes.
  \item Promedio de P.A.P.I de los estudiantes
  \item Desviación estandar de P.A.P.I de los estudiantes
  \item Máximo número de materias que un estudiante va a inscribir (mayor o igual a 3, 
        menor o igual a lista de deseos).
  \item Máximo número de materias deseadas de un estudiante (Predeterminado=9, mayor o igual a 3).
  \item Tiempo de simulación
\end{itemize}

Todos los parámetros de entrada son opcionales.


%---------------------------------------------------------------------------------
% Manual Técnico ---------------------------------------------------------
%---------------------------------------------------------------------------------

\section{Manual Técnico}\label{sec:man_t}


%---------------------------------------------------------------------------------
% Experimentación ---------------------------------------------------------
%---------------------------------------------------------------------------------

\section{Experimentación}\label{sec:exp}

\subsection{Análisis de resultados}

\subsubsection{Escenario 1: Asignación Aleatoria Estándar (PS)}
En este primer escenario, todos los estudiantes participan simultáneamente en un único
 proceso de asignación mediante el \emph{Probabilistic Serial Rule}. No se aplica 
 ninguna modificación a la distribución de cupos ni se altera el orden de inscripción.
 El orden de inscripción se determina aleatoriamente y se mantiene constante a lo largo
  del proceso.

\begin{itemize}
\item Se instancian todos los estudiantes y todas las clases con sus cupos originales.
\item Se ejecuta el algoritmo PS completo (fase continua de consumo fraccional y 
redondeo greedy).
\item Se registran las asignaciones finales de cada estudiante en escenario1\_estudiantes.json y 
 el estado de cupos remanentes en escenario1\_clases.json.
\end{itemize}

Este escenario sirve como línea base para comparar los efectos de políticas de prioridad
 y cancelacion en los siguientes experimentos.

 \subsubsection{Escenario 2: Asignación con Prioridad Académica (P.A.P.I)}
 En este escenario, todos los estudiantes participan en un único proceso de asignación mediante el
  \emph{Probabilistic Serial Rule}, pero el orden de inscripción se determina por el índice académico 
  (p.a.p.i) en lugar de ser aleatorio. Los estudiantes con mayor P.A.P.I obtienen prioridad para 
  consumir cupos fraccionales durante todo el proceso.
 
 \begin{itemize}
 \item Se instancian todos los estudiantes y clases con cupos originales.
 \item Se ordenan los estudiantes de forma descendente según su p.a.p.i antes de iniciar el algoritmo.
 \item Se ejecuta el algoritmo PS completo manteniendo este orden prioritario durante ambas fases (consumo fraccional y redondeo).
 \item Durante el redondeo greedy, se respeta el orden académico para asignaciones enteras en caso de empates.
 \end{itemize}
 
 Este escenario evalúa cómo la priorización por mérito académico afecta la distribución de cupos comparado con el modelo base aleatorio.

\subsubsection{Escenario 3: Asignación en Dos Fases con Grupo Prioritario (20\%)}
En este escenario se evalúa el impacto de otorgar prioridad de inscripción a un subconjunto 
específico de estudiantes. Para ello, se selecciona aleatoriamente un grupo que representa el 
20\% del total de estudiantes. Este grupo realiza primero el proceso de asignación de materias 
mediante el mecanismo base (PS) y su respectivo redondeo.

El 80\% restante de estudiantes entra al proceso de asignación únicamente una vez que el grupo 
prioritario ha completado su selección y ya no tiene materias adicionales por consumir. 
De este modo, se simula una política de inscripción diferenciada en dos etapas, en la que se 
garantiza prioridad efectiva para un subconjunto determinado.


\begin{itemize}
\item El grupo prioritario ejecuta primero el PS completo con cupos originales.
\item Se actualizan los cupos de clases después de la fase prioritaria.
\item El grupo regular ejecuta PS solo sobre los cupos remanentes.
\item Se combinan los resultados de ambas fases preservando las asignaciones prioritarias.
\end{itemize}

Este diseño permite analizar los efectos de crear ventanas prioritarias en el acceso a cupos.

\subsubsection{Escenario 4: Asignación con Cancelación Temporal de Clases}

Este escenario aborda el problema de cancelaciones recurrentes en asignaturas como por ejemplo las de libre ellección
 mediante un proceso estructurado de dos fases. Se busca seleccionar las clases con mayor tasa de cancelación
  y excluirlas temporalmente de una primera fase de inscripción. Los estudiantes asignan primero cupos en asignaturas básicas/obligatorias
   (con menor tasa de cancelación),
  estableciendo una base académica sólida, antes de inscribir electivas en una segunda fase. 

Para simular este comportamiento se plantea una interrupción temporal donde el 30\% de las clases se marcan 
como no disponibles durante la fase de consumo fraccional, 
pero se restauran antes del redondeo final. 

\begin{itemize}
\item Se seleccionan aleatoriamente el 30\% de clases para "cancelar" temporalmente (cupo = 0).
\item Se ejecuta la fase de consumo fraccional de PS con esta oferta reducida.
\item Se restauran los cupos originales de las clases canceladas.
\item Se realiza el redondeo greedy usando los consumos registrados durante la interrupción.
\item Se registran las asignaciones resultantes y se comparan contra el escenario base.
\end{itemize}

\subsubsection{Comparación resultados (30\%)}

Se evaluaron los cuatro escenarios utilizando seis métricas clave para analizar la calidad de las asignaciones:

\begin{itemize}
  \item \textbf{Satisfacción media:} promedio del grado de satisfacción de los estudiantes según la prioridad de asignación (más alta si obtienen su primera opción).
  \item \textbf{Uso de cupos:} proporción de utilización de los cupos disponibles en los cursos.
  \item \textbf{Porcentaje de asignación por opción:} incluye la proporción de estudiantes asignados a su primera, segunda y tercera opción, así como el porcentaje que obtuvo alguna de las tres.
  \item \textbf{Equidad:} medida de justicia en la distribución de las asignaciones entre los estudiantes.
  \item \textbf{Eficiencia de Pareto:} porcentaje de asignaciones que no podrían mejorar para un estudiante sin empeorar para otro.
\end{itemize}

A continuación, se presentan los promedios de cada métrica en 20 iteraciones por escenario:

\begin{table}[H]
\centering
\caption{Promedios por escenario}
\begin{tabular}{lcccc}
\toprule
\textbf{Métrica} & \textbf{Escenario 1} & \textbf{Escenario 2} & \textbf{Escenario 3} & \textbf{Escenario 4} \\
\midrule
Satisfacción media & 0.638 & 0.638 & \textbf{0.659} & 0.533 \\
Uso de cupos & 0.717 & 0.717 & \textbf{0.742} & 0.598 \\
Primera opción (\%) & \textbf{99.73} & 99.72 & 99.70 & 66.00 \\
Segunda opción (\%) & 51.11 & 51.13 & \textbf{53.16} & 44.37 \\
Tercera opción (\%) & 33.24 & 33.24 & \textbf{35.19} & 32.00 \\
Alguna de las tres (\%) & 15.49 & 15.53 & \textbf{16.89} & 4.46 \\
Equidad & 0.158 & \textbf{0.158} & 0.156 & \textbf{0.221} \\
Eficiencia de Pareto & 0.0655 & 0.0656 & \textbf{0.0712} & 0.0182 \\
\bottomrule
\end{tabular}
\end{table}

\noindent Al analizar los resultados obtenidos, se pueden identificar diferencias notables en el comportamiento de cada política de asignación implementada en los escenarios.

El \textbf{Escenario 2}, que incorpora prioridad académica mediante el índice P.A.P.I., presenta el mejor desempeño general en términos de satisfacción media, eficiencia de Pareto y aprovechamiento de cupos. Esto se explica porque los estudiantes con mejor rendimiento académico acceden primero a los cupos más demandados, lo que tiende a maximizar la utilidad agregada del sistema. La satisfacción media alcanza un valor de 0.659, con altos porcentajes de asignación en la segunda y tercera opción, lo cual sugiere que incluso los estudiantes sin prioridad directa logran acceder a opciones razonablemente cercanas a sus preferencias. Adicionalmente, la eficiencia de Pareto más alta (0.0712) indica que las asignaciones realizadas presentan un menor margen para mejoras sin perjudicar a otros, consolidando este escenario como el más eficiente desde una perspectiva global.

En contraste, el \textbf{Escenario 3}, que introduce una política de prioridad a un subconjunto aleatorio (20\% de los estudiantes), muestra una importante mejora en la métrica de \textit{equidad} (0.221), lo que indica una distribución más homogénea de la satisfacción. Esta política logra reducir la disparidad entre estudiantes priorizados y no priorizados, pero a un costo significativo: la satisfacción media se reduce a 0.533, el uso de cupos cae al 59.8\%, y la eficiencia de Pareto se desploma a 0.0182. Además, solo el 66\% de los estudiantes obtiene su primera opción, lo cual contrasta fuertemente con los más del 99\% observados en los otros escenarios. Esto revela que una política de ventanas prioritarias puede aumentar la justicia percibida, pero sacrifica notoriamente la eficiencia y la calidad general de la asignación.

El \textbf{Escenario 1}, que utiliza una asignación aleatoria estándar mediante el algoritmo PS, sirve como línea base y ofrece un balance moderado entre eficiencia y equidad. Sin intervenciones adicionales, mantiene una satisfacción media aceptable (0.635), aunque inferior al Escenario 2. La asignación a las primeras opciones se distribuye de manera más neutral, y el uso de cupos y la eficiencia de Pareto se mantienen en niveles razonables. Este enfoque representa una política neutral que no privilegia ni penaliza a ningún grupo, siendo útil como referencia para medir el impacto de las demás estrategias.

Por último, el \textbf{Escenario 4}, que introduce una cancelación temporal del 30\% de clases (típicamente electivas), se sitúa en una posición intermedia. Su objetivo de priorizar asignaturas básicas busca evitar cancelaciones y asegurar trayectorias académicas sólidas. Aunque no alcanza los niveles de satisfacción ni eficiencia del Escenario 2, presenta un desempeño razonable con una satisfacción media de 0.630 y un uso de cupos del 71.9\%. La eficiencia de Pareto también es considerablemente mejor que la del Escenario 3. Este enfoque parece balancear adecuadamente las restricciones académicas institucionales con las preferencias de los estudiantes, y puede considerarse una estrategia viable para contextos con alta inestabilidad en la oferta académica.

En resumen, el Escenario 2 sobresale en términos de eficiencia y satisfacción, ideal para contextos donde se prioriza el mérito académico. El Escenario 3 destaca por su enfoque en la equidad, pero con un fuerte sacrificio en eficiencia. El Escenario 1 sirve como punto de referencia neutral, mientras que el Escenario 4 ofrece una alternativa con compromisos interesantes entre estructura institucional y bienestar estudiantil.


\section{Referencias}
\renewcommand{\refname}{}
\begin{thebibliography}{9}

\bibitem{ref} \label{ref:BPS} M. Bichler, S. Merting, and A. Uzunoglu, 
“Assigning Course Schedules: About Preference Elicitation, Fairness, and Truthfulness,” 
arXiv preprint arXiv:1812.02630, 2018. [En línea]. Disponible en: 
\url{https://arxiv.org/abs/1812.02630}

\bibitem{ref} \label{ref:MatchingBundle}
M. Bichler y S. Merting, ``Matching with Bundle Preferences: Tradeoff between Fairness 
and Truthfulness,'' en \textit{Proceedings of the 14th International Conference on 
Wirtschaftsinformatik}, Siegen, Alemania, 2019. [En línea]. Disponible en: 
\url{https://aisel.aisnet.org/wi2019/track05/papers/1/}

\bibitem{ref} \label{ref:RSD1} 
A. Bogomolnaia y H. Moulin, 
“A new solution to the random assignment problem,” 
\textit{Journal of Economic Theory}, vol. 100, no. 2, pp. 295–328, 2001. [En línea]. Disponible en: \url{https://doi.org/10.1006/jeth.2000.2723}

\bibitem{ref} \label{ref:RSD2} 
A. Abdulkadiroğlu y T. Sönmez, 
“Random serial dictatorship and the core from random endowments in house allocation problems,” 
\textit{Econometrica}, vol. 66, no. 3, pp. 689–701, 1998. [En línea]. Disponible en: \url{https://doi.org/10.2307/2998575}

\bibitem{ref} \label{ref:PBS} A. Bogomolnaia and H. Moulin, “A new solution to the random 
assignment problem,” Journal of Economic Theory, vol. 100, no. 2, pp. 295–328, 2001, doi: 
10.1006/jeth.2000.2710.

\bibitem{ref} \label{ref:Mercados}
D.~Kornbluth y A.~Kushnir, ``Undergraduate Course Allocation through Competitive Markets,'' \textit{SSRN Electronic Journal}, 2023. [En línea]. Disponible en: \url{https://papers.ssrn.com/sol3/papers.cfm?abstract_id=3901146}

\bibitem{ref:Kornbluth2023} \label{ref:AprendizajeAutomatico} 
D. Kornbluth y A. Kushnir, 
\textit{“Machine Learning-powered Course Allocation”}, 
Carnegie Mellon University / Harvard University, 2023. [En línea]. Disponible en: \url{https://www.andrew.cmu.edu/user/dkornblu/CourseAllocation.pdf}


\end{thebibliography}

\end{document}