\documentclass{article}
\usepackage{graphicx}
\usepackage[style=ieee]{biblatex} % Establecer el estilo de las referencias como IEEE
\usepackage{xcolor}
\usepackage{hyperref}
\usepackage{titletoc}
\usepackage{adjustbox}
\usepackage{amsmath}
\usepackage[spanish]{babel}

\usepackage{listings}
\hypersetup{
    colorlinks=true,
    linkcolor=blue, % Color del texto del enlace
    urlcolor=blue % Color del enlace
}

\usepackage{longtable} % Agrega el paquete longtable

\definecolor{mygreen}{RGB}{0,128,0}

\usepackage{array} % Para personalizar la tabla
\usepackage{booktabs} % Para líneas horizontales de mejor calidad
\usepackage{graphicx} % Paquete para incluir imágenes
\usepackage{float}

% Definir márgenes
\usepackage[margin=1in]{geometry}

\renewcommand{\contentsname}{\textcolor{mygreen}{Tabla de Contenidos}}

\begin{document}

\begin{titlepage}
    \centering
    % Logo de la Universidad
    \includegraphics[width=0.48\textwidth]{logo_universidad.png}
    \par\vspace{2cm}

    % Nombre de la Universidad y detalles del curso
    {\Large \textbf{Universidad Nacional de Colombia} \par}
    \vspace{0.5cm}
    {\large Ingeniería de Sistemas y Computación \par}
    {\large 2025969 Modelos estocásticos y simulación en computación y comunicaciones (01)\par}
    \vspace{3cm}

    % Detalles del laboratorio y actividad
    {\large \textbf{Taller 1} \par}
    \vspace{3cm}

    % Lista de integrantes
    {\large \textbf{Integrantes:} \par}
    \vspace{0.5cm}
    \begin{tabular}{ll}
    Javier Andrés Tarazona Jiménez & jtarazonaj@unal.edu.co \\
    Yenifer Yulieth Mora Segura & ymoras@unal.edu.co \\
    Juan Esteban Carranza Salazar & jcarranza@unal.edu.co \\
    Grevy Joner Rincon Mejia & grrinconm@unal.edu.co \\
    Jefferson Duvan Ramirez Castañeda & jeramirezca@unal.edu.co \\
    Javier Andres Carrillo Carrasco & jacarrillo@unal.edu.co \\
    Diego Nicolas Ramirez Maldonado & dieramirezma@unal.edu.co \\
    \end{tabular}
    \par\vspace{3cm}

    % Fecha
    {\large Julio 13 de 2025 \par}
\end{titlepage}

\tableofcontents % Inserta la tabla de contenidos

\newpage % Salto de página para separar la tabla de contenidos del contenido del documento

% Contenido del artículo----------------------------------------------------------

%---------------------------------------------------------------------------------
% Intro --------------------------------------------------------------------------
%---------------------------------------------------------------------------------

\section{Introducción}\label{sec:intr}


%---------------------------------------------------------------------------------
% Marco Teórico ------------------------------------------------------------------
%---------------------------------------------------------------------------------

\section{Marco Teórico}\label{sec:marc}


%---------------------------------------------------------------------------------
% Descr. Problema ------------------------------------------------------------------
%---------------------------------------------------------------------------------

\section{Descripción y Justificación del Problema a Resolver}\label{sec:descr}

\subsection{Contexto del Problema}
En el estudio de sistemas estocásticos, los modelos de colas son herramientas esenciales para analizar procesos donde la demanda de recursos supera su disponibilidad, como en redes de telecomunicaciones, sistemas logísticos o servicios al cliente. El modelo M/M/1 (llegadas y servicios Markovianos con un servidor) es un caso fundamental para entender métricas como tiempo de espera, longitud de cola y ocupación del sistema. Sin embargo, su aplicación práctica requiere:
\begin{enumerate}
    \item Implementaciones computacionales precisas (como el simulador en C++ proporcionado, basado en la sección 1.4.4 del libro de Law).
    \item Validación rigurosa mediante comparación con resultados teóricos y datos empíricos (hoja de cálculo adjunta).
    \item Extensibilidad para abordar modelos más complejos (ej: colas Geom/Geom/m/N o fórmulas de Erlang).
\end{enumerate}

Los recursos proporcionados (Law, 2014; Ortiz, 2023) ofrecen el marco teórico y metodológico para cumplir estos requisitos, mientras que el simulador y los datos empíricos permiten la aplicación práctica.

\subsection{Problema Específico}

Se identifican tres brechas principales:

\begin{itemize}
    \item \textbf{Implementación inicial incompleta}: El simulador requiere verificación según los 6 algoritmos de \cite{Ortiz2023} (Sección 1.7.1).
    \item \textbf{Validación insuficiente}: Los resultados deben coincidir con datos empíricos y fórmulas teóricas \cite{Law2014}.
    \item \textbf{Limitaciones funcionales}: Falta soporte para modelos avanzados (Geom/Geom/m/N y fórmulas de Erlang).
\end{itemize}

\subsection{Objetivo Principal}

Desarrollar un simulador mejorado que:
\begin{enumerate}
    \item Sea modular (6 algoritmos independientes)
    \item Valide resultados con datos empíricos
    \item Extienda funcionalidades para:
    \begin{itemize}
        \item Fórmulas B y C de Erlang
        \item Modelos Geom/Geom/m/N
    \end{itemize}
    \item Garantice reproducibilidad
\end{enumerate}

%\subsection{Hipótesis de Investigación}

\subsection{Justificación de la Solución Propuesta}

La solución se justifica por:
\begin{itemize}
    \item \textbf{Rigor académico}: Sigue estándares de \cite{Law2014} y \cite{Ortiz2023}
    \item \textbf{Validez práctica}: Comparación con datos reales y teoría
    \item \textbf{Aplicabilidad ampliada}: Modelos avanzados para escenarios realistas
    \item \textbf{Impacto formativo}: Refuerza competencias en simulación
\end{itemize}

\subsection{Métricas de Evaluación}


%---------------------------------------------------------------------------------
% Diseño solución ------------------------------------------------------------------
%---------------------------------------------------------------------------------


\section{Diseño de la solución}\label{sec:disSol}


%---------------------------------------------------------------------------------
% Código Fuente ------------------------------------------------------------------
%---------------------------------------------------------------------------------

\section{Código Fuente}\label{sec:cod}

El código fuente completo se encuentra adjunto como Taller1.zip
o en el siguiente repositorio de GitHub:

\begin{center}
\url{https://github.com/JavierTarazona06/ME01_Tareas/tree/main/taller2/Code/}
\end{center}


%---------------------------------------------------------------------------------
% Manual de Usuario ------------------------------------------------------------------
%---------------------------------------------------------------------------------

\section{Manual Usuario}\label{sec:man_u}


%---------------------------------------------------------------------------------
% Manual técnico ------------------------------------------------------------------
%---------------------------------------------------------------------------------

\section{Manual Técnico}


%---------------------------------------------------------------------------------
% Exprimentación ------------------------------------------------------------------
%---------------------------------------------------------------------------------


\section{Experimentación}\label{sec:exp}



%---------------------------------------------------------------------------------
% Conclusiones ---------------------------------------------------------
%---------------------------------------------------------------------------------

\section{Conclusiones}\label{sec:concl}


%---------------------------------------------------------------------------------
% Recomendaciones ---------------------------------------------------------
%---------------------------------------------------------------------------------

\section{Recomendaciones}\label{secrecomen}





\section{Referencias}
\renewcommand{\refname}{}
\begin{thebibliography}{9}

\bibitem{ref} \label{ref:modSim} A. M. Law, *Simulation Modeling and Analysis*, 5th ed. 
New York, NY, USA: McGraw-Hill Education, Jan. 2014, ISBN 978-0-07-340132-4. 

\bibitem{ref} \label{ref:matSim} J. E. Ortiz T., “Modelos Matemáticos \& Simulación,”
 en *Apuntes de clase de la asignatura Modelos Estocásticos y Simulación en 
 Computación y Comunicaciones*, Departamento de Ingeniería de Sistemas e Industrial, 
 Universidad Nacional de Colombia, 2025. [En línea]. Disponible en: 
 \url{https://drive.google.com/file/d/1c886yU4SFk9A97DGWr8JHiF3nd2au1Ns/view?usp=sharing}

 \bibitem{ref} \label{ref:cimColas} J. E. UOrtiz, “Simulador del modelo de colas M/M/1,” 
 código fuente en C++ (archivo .rar), basado en la sección 
 1.4.4 de Averill M. Law, *Simulation Modeling and Analysis*, 
 th ed., McGraw-Hill, 2014. [En línea]. Disponible en: 
 \url{https://drive.google.com/file/d/1-Mo9hqPwOegwTpx0KP2tbD9sLTNSXefd/view?usp=sharing}



\end{thebibliography}

\end{document}