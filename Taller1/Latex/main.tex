\documentclass{article}
\usepackage{graphicx}
\usepackage[style=ieee]{biblatex} % Establecer el estilo de las referencias como IEEE
\usepackage{xcolor}
\usepackage{hyperref}
\usepackage{titletoc}
\usepackage{adjustbox}

\hypersetup{
    colorlinks=true,
    linkcolor=blue, % Color del texto del enlace
    urlcolor=blue % Color del enlace
}

\usepackage{longtable} % Agrega el paquete longtable

\definecolor{mygreen}{RGB}{0,128,0}

\usepackage{array} % Para personalizar la tabla
\usepackage{booktabs} % Para líneas horizontales de mejor calidad
\usepackage{graphicx} % Paquete para incluir imágenes
\usepackage{float}

% Definir márgenes
\usepackage[margin=1in]{geometry}

\renewcommand{\contentsname}{\textcolor{mygreen}{Tabla de Contenidos}}

\begin{document}

\begin{titlepage}
    \centering
    % Logo de la Universidad
    \includegraphics[width=0.48\textwidth]{logo_universidad.png}
    \par\vspace{2cm}

    % Nombre de la Universidad y detalles del curso
    {\Large \textbf{Universidad Nacional de Colombia} \par}
    \vspace{0.5cm}
    {\large Ingeniería de Sistemas y Computación \par}
    {\large 2025969 Modelos estocásticos y simulación en computación y comunicaciones (01)\par}
    \vspace{3cm}

    % Detalles del laboratorio y actividad
    {\large \textbf{Taller 1} \par}
    {\large Taller 1 \par}
    \vspace{3cm}

    % Lista de integrantes
    {\large \textbf{Integrantes:} \par}
    \vspace{0.5cm}
    \begin{tabular}{ll}
    Javier Andrés Tarazona Jiménez & jtarazonaj@unal.edu.co \\
    Jefferson Duvan Ramirez Castañeda & jeramirezca@unal.edu.co \\
    Yenifer Yulieth Mora Segura & ymoras@unal.edu.co \\
    Javier Carrillo & @unal.edu.co \\
    Grevy Joner Rincon Mejia & grrinconm@unal.edu.co \\
    Esteban Carranza & jcarranza@unal.edu.co \\
    Diego & @unal.edu.co \\
    \end{tabular}
    \par\vspace{3cm}

    % Fecha
    {\large Junio 7 de 2025 \par}
\end{titlepage}

\tableofcontents % Inserta la tabla de contenidos

\newpage % Salto de página para separar la tabla de contenidos del contenido del documento

% Contenido del artículo----------------------------------------------------------

%---------------------------------------------------------------------------------
% Intro --------------------------------------------------------------------------
%---------------------------------------------------------------------------------

\section{Introducción}\label{sec:intr}

En un escenario de extinción de incendios forestales mediante drones cooperativos, la comunicación inalámbrica entre los dispositivos se ve constantemente afectada por obstáculos naturales (árboles, relieve), interferencias y la movilidad dinámica del enjambre. Este trabajo aborda un problema concreto: cuando un dron se desconecta momentáneamente del líder de su grupo (Cluster Head) debido a estas adversidades, pierde capacidad de coordinación crítica para la misión, retrasando la respuesta ante focos de incendio y aumentando el consumo energético por retransmisiones.

Para resolver este desafío operativo, proponemos una solución basada en redes MANET jerárquicas, donde los drones desconectados adoptan temporalmente el rol de líder (CH-temporal). Este enfoque práctico se sustenta en modelos teóricos de movilidad grupal y algoritmos de reclusterización dinámica, evaluando umbrales de desconexión basados en calidad de señal, energía remanente y presencia de obstáculos. La implementación en NS-3 demuestra que este método reduce el tiempo de recuperación de conexiones comparado con protocolos tradicionales, manteniendo la efectividad del enjambre incluso en condiciones adversas.


%---------------------------------------------------------------------------------
% Marco Teórico ------------------------------------------------------------------
%---------------------------------------------------------------------------------

\section{Marco Teórico}\label{sec:marc}

\section{Descripción y Justificación del Problema a Resolver}\label{sec:descr}


\subsection{Objetivo Principal}

%---------------------------------------------------------------------------------
% Diseño de la solución ---------------------------------------------------------
%---------------------------------------------------------------------------------

\section{Diseño de la solución}\label{sec:dis}

Para abordar la solución, se presentan las áreas y abstracciones que se van a abordar.

\subsection{Metodología}
\subsection{Componentes del Sistema}
\subsection{Flujo de ejecución}


\section{Código Fuente}\label{sec:cod}

El código fuente completo de este modelo se encuentra adjunto en el buzón 
(06 Tarazona Jimenez Javier Andres 02.zip)
y disponible en el repositorio GitHub del proyecto:

\begin{center}
\url{https://github.com/JavierTarazona06/ME01_Tareas/tree/main/Tarea06/Code}
\end{center}

El repositorio contiene:
\begin{itemize}
\item Los módulos principales para generación de datos (\texttt{generator.py})
\item Implementación del algoritmo PBS (\texttt{toolsStats.py})
\item Scripts de simulación para los 4 escenarios, ejecutables desde (\texttt{main.py})
\item Archivo program.py de para la definición de las constantes del programa 
    (\texttt{constants\_program.py})
\item Archivo de métricas (\texttt{metrics.py})
\end{itemize}

%---------------------------------------------------------------------------------
% Manual Usuario ---------------------------------------------------------
%---------------------------------------------------------------------------------

\section{Manual Usuario}\label{sec:man_u}


%---------------------------------------------------------------------------------
% Manual Técnico ---------------------------------------------------------
%---------------------------------------------------------------------------------

\section{Manual Técnico}\label{sec:man_t}



\section{Experimentación}\label{sec:exp}

\subsection{Documentación}


\subsection{Escenario 1:}


\subsection{Escenario 2:}

\subsection{Escenario 3:}

\subsection{Comparación resultados}

%---------------------------------------------------------------------------------
% Conclusiones ---------------------------------------------------------
%---------------------------------------------------------------------------------

\section{Conclusiones}\label{sec:concl}

%---------------------------------------------------------------------------------
% Recomendaciones ---------------------------------------------------------
%---------------------------------------------------------------------------------

\section{Recomendaciones}\label{secrecomen}


\section{Referencias}
\renewcommand{\refname}{}
\begin{thebibliography}{9}

\bibitem{ref} \label{ref:BPS} M. Bichler, S. Merting, and A. Uzunoglu, 
“Assigning Course Schedules: About Preference Elicitation, Fairness, and Truthfulness,” 
arXiv preprint arXiv:1812.02630, 2018. [En línea]. Disponible en: 
\url{https://arxiv.org/abs/1812.02630}



\end{thebibliography}

\end{document}