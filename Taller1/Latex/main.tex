\documentclass{article}
\usepackage{graphicx}
\usepackage[style=ieee]{biblatex} % Establecer el estilo de las referencias como IEEE
\usepackage{xcolor}
\usepackage{hyperref}
\usepackage{titletoc}
\usepackage{adjustbox}

\hypersetup{
    colorlinks=true,
    linkcolor=blue, % Color del texto del enlace
    urlcolor=blue % Color del enlace
}

\usepackage{longtable} % Agrega el paquete longtable

\definecolor{mygreen}{RGB}{0,128,0}

\usepackage{array} % Para personalizar la tabla
\usepackage{booktabs} % Para líneas horizontales de mejor calidad
\usepackage{graphicx} % Paquete para incluir imágenes
\usepackage{float}

% Definir márgenes
\usepackage[margin=1in]{geometry}

\renewcommand{\contentsname}{\textcolor{mygreen}{Tabla de Contenidos}}

\begin{document}

\begin{titlepage}
    \centering
    % Logo de la Universidad
    \includegraphics[width=0.48\textwidth]{logo_universidad.png}
    \par\vspace{2cm}

    % Nombre de la Universidad y detalles del curso
    {\Large \textbf{Universidad Nacional de Colombia} \par}
    \vspace{0.5cm}
    {\large Ingeniería de Sistemas y Computación \par}
    {\large 2025969 Modelos estocásticos y simulación en computación y comunicaciones (01)\par}
    \vspace{3cm}

    % Detalles del laboratorio y actividad
    {\large \textbf{Taller 1} \par}
    {\large Taller 1 \par}
    \vspace{3cm}

    % Lista de integrantes
    {\large \textbf{Integrantes:} \par}
    \vspace{0.5cm}
    \begin{tabular}{ll}
    Javier Andrés Tarazona Jiménez & jtarazonaj@unal.edu.co \\
    Jefferson Duvan Ramirez Castañeda & jeramirezca@unal.edu.co \\
    Yenifer Yulieth Mora Segura & ymoras@unal.edu.co \\
    Javier Carrillo & @unal.edu.co \\
    Grevy Joner Rincon Mejia & grrinconm@unal.edu.co \\
    Esteban Carranza & jcarranza@unal.edu.co \\
    Diego & @unal.edu.co \\
    \end{tabular}
    \par\vspace{3cm}

    % Fecha
    {\large Junio 7 de 2025 \par}
\end{titlepage}

\tableofcontents % Inserta la tabla de contenidos

\newpage % Salto de página para separar la tabla de contenidos del contenido del documento

% Contenido del artículo----------------------------------------------------------

%---------------------------------------------------------------------------------
% Intro --------------------------------------------------------------------------
%---------------------------------------------------------------------------------

\section{Introducción}\label{sec:intr}

En un escenario de extinción de incendios forestales mediante drones cooperativos, la comunicación inalámbrica entre los dispositivos se ve constantemente afectada por obstáculos naturales (árboles, relieve), interferencias y la movilidad dinámica del enjambre. Este trabajo aborda un problema concreto: cuando un dron se desconecta momentáneamente del líder de su grupo (Cluster Head) debido a estas adversidades, pierde capacidad de coordinación crítica para la misión, retrasando la respuesta ante focos de incendio y aumentando el consumo energético por retransmisiones.

Para resolver este desafío operativo, proponemos una solución basada en redes MANET jerárquicas, donde los drones desconectados adoptan temporalmente el rol de líder (CH-temporal). Este enfoque práctico se sustenta en modelos teóricos de movilidad grupal y algoritmos de reclusterización dinámica, evaluando umbrales de desconexión basados en calidad de señal, energía remanente y presencia de obstáculos. La implementación en NS-3 demuestra que este método reduce el tiempo de recuperación de conexiones comparado con protocolos tradicionales, manteniendo la efectividad del enjambre incluso en condiciones adversas.


%---------------------------------------------------------------------------------
% Marco Teórico ------------------------------------------------------------------
%---------------------------------------------------------------------------------

\section{Marco Teórico}\label{sec:marc}

\section{Descripción y Justificación del Problema a Resolver}\label{sec:descr}


\subsection{Objetivo Principal}

%---------------------------------------------------------------------------------
% Diseño de la solución ---------------------------------------------------------
%---------------------------------------------------------------------------------



\section{Diseño de la solución}

La solución desarrollada está basada en el modelo de comportamiento de Boids para simular la movilidad de nodos en una red ad hoc, adaptando las reglas clásicas de \textit{cohesión}, \textit{alineación} y \textit{separación}, e incorporando elementos adicionales como nodos líderes, mecanismos de selección de líderes mediante el algoritmo de clustering ponderado (WCA) y la influencia de eventos externos, como incendios, que afectan la dirección de los nodos.

\subsection{Metodología}

El desarrollo siguió un enfoque incremental: primero se construyó la jerarquía básica de nodos (líderes y seguidores), luego se creo la logica de control de red, se integraron de forma progresiva con las funcionalidades dinámicas estocásticas y el contexto de la simulacion, y finalmente se evaluaron distintos escenarios con variaciones de parámetros clave.

\subsection{Estructura general}

La clase \texttt{BoidsMobilityModel} extiende el modelo de movilidad de NS-3 y contiene la lógica de comportamiento individual de cada nodo. Cada nodo se comporta como un Boid, actualizando su posición periódicamente en función de su rol (líder o seguidor) y de su entorno.

\subsection{Reglas de comportamiento}

\begin{itemize}
    \item \textbf{Separación:} Los nodos evitan colisionar entre sí manteniéndose alejados de los vecinos más cercanos dentro de un radio definido.
    
    \item \textbf{Alineación:} Los nodos no líderes ajustan su dirección promedio en función de los vectores de velocidad de sus vecinos.
    
    \item \textbf{Cohesión:} Los seguidores intentan mantenerse cerca de un líder si este se encuentra dentro de un cierto radio de influencia.
    
    \item \textbf{Atracción hacia objetivos:} Los líderes se ven atraídos hacia eventos externos (como incendios) si estos se encuentran dentro de su radio de influencia.
\end{itemize}

\subsection{Liderazgo dinámico y WCA}

Cada nodo puede actuar como líder o seguidor. El rol inicial se define al inicio de la simulación, y a los líderes se les asigna un objetivo aleatorio en el espacio. Si un nodo seguidor se encuentra aislado, es decir, sin un líder dentro de su radio de cohesión, evalúa su potencial de liderazgo utilizando el algoritmo \textit{WCA} (Weighted Clustering Algorithm). Este cálculo considera los siguientes parámetros:

\begin{itemize}
    \item \textit{Energía residual} (\(E\)): cantidad de energía restante, normalizada en el rango \([0,1]\).
    \item \textit{Grado de conectividad} (\(D\)): número de vecinos directos, normalizado considerando un máximo de 10.
    \item \textit{Distancia a los objetivos} (\(T\)): inversa de la distancia promedio a los objetivos, considerando un radio de 200 metros.
    \item \textit{Movilidad} (\(M\)): inversa de la velocidad, considerando una velocidad máxima de 10 m/s.
\end{itemize}

Cada uno de estos parámetros se normaliza para quedar en el rango \([0,1]\), y luego se combinan mediante una fórmula ponderada. La métrica de liderazgo (\texttt{WCA\_Score}) se calcula así:

\[
{WCA\_Score} = w_1 \cdot E + w_2 \cdot D + w_3 \cdot T + w_4 \cdot M
\]
donde los pesos \(w_1, w_2, w_3, w_4\) determinan la importancia relativa de cada factor y cumplen que:

\[
w_1 = 0.4, \quad w_2 = 0.3, \quad w_3 = 0.2, \quad w_4 = 0.1
\]

Finalmente, el puntaje se acota al intervalo \([0, 1]\):

\[
{WCA\_Score} = \max\left(0,\ \min\left(1,\ {WCA\_Score}\right)\right)
\]
\subsection{Espacio de simulación y condiciones de frontera}

El espacio simulado tiene dimensiones de $1000 \times 1000$ unidades. Se utiliza un modelo de \textit{wrapping} para simular un entorno toroidal, permitiendo que los nodos que cruzan un borde reaparezcan en el lado opuesto. Esto previene la acumulación de nodos en los bordes y garantiza una distribución continua.

\subsection{Actualización de posiciones}

Cada nodo actualiza su posición en cada intervalo de tiempo simulando su movimiento. Se aplican límites a la velocidad máxima y se normalizan los vectores de dirección para evitar aceleraciones irregulares. Las velocidades también se ajustan para evitar que los nodos queden inmóviles.

\subsection{Visualización y exportación de datos}

Al finalizar la simulación, se exportan los datos relevantes a un archivo de salida. Esta información incluye la posición de los nodos, su rol (líder o seguidor), y la ubicación de eventos externos como incendios. Estos datos permiten un análisis posterior del comportamiento emergente del sistema.

\subsection{Configurabilidad}

Se incluyen métodos \texttt{Set} para ajustar los parámetros más relevantes del modelo:
\begin{itemize}
    \item Radio de cohesión
    \item Radio de influencia de líderes
    \item Velocidad máxima
    \item Definición inicial del rol de liderazgo
\end{itemize}

Esto permite ejecutar múltiples experimentos modificando dinámicamente los parámetros del modelo, facilitando su validación bajo distintos escenarios.

\subsection{Simulación de escenarios}

Finalmente, con el fin de evaluar el impacto de permitir que un nodo seguidor aislado se autoproclame temporalmente como líder, se diseñaron tres escenarios de simulación. Cada uno busca analizar cómo esta decisión afecta la efectividad global en la extinción de fuegos, medida principalmente a través del tiempo promedio de extinción y el número total de fuegos extinguidos en un período fijo.


\section{Código Fuente}\label{sec:cod}

El código fuente completo de este modelo se encuentra adjunto en el buzón 
(06 Tarazona Jimenez Javier Andres 02.zip)
y disponible en el repositorio GitHub del proyecto:

\begin{center}
\url{https://github.com/JavierTarazona06/ME01_Tareas/tree/main/Tarea06/Code}
\end{center}

El repositorio contiene:
\begin{itemize}
\item Los módulos principales para generación de datos (\texttt{generator.py})
\item Implementación del algoritmo PBS (\texttt{toolsStats.py})
\item Scripts de simulación para los 4 escenarios, ejecutables desde (\texttt{main.py})
\item Archivo program.py de para la definición de las constantes del programa 
    (\texttt{constants\_program.py})
\item Archivo de métricas (\texttt{metrics.py})
\end{itemize}

%---------------------------------------------------------------------------------
% Manual Usuario ---------------------------------------------------------
%---------------------------------------------------------------------------------

\section{Manual Usuario}\label{sec:man_u}


%---------------------------------------------------------------------------------
% Manual Técnico ---------------------------------------------------------
%---------------------------------------------------------------------------------

\section{Manual Técnico}\label{sec:man_t}



\section{Experimentación}\label{sec:exp}

\subsection{Documentación}


\subsection{Escenario 1:}


\subsection{Escenario 2:}

\subsection{Escenario 3:}

\subsection{Comparación resultados}

%---------------------------------------------------------------------------------
% Conclusiones ---------------------------------------------------------
%---------------------------------------------------------------------------------

\section{Conclusiones}\label{sec:concl}

%---------------------------------------------------------------------------------
% Recomendaciones ---------------------------------------------------------
%---------------------------------------------------------------------------------

\section{Recomendaciones}\label{secrecomen}


\section{Referencias}
\renewcommand{\refname}{}
\begin{thebibliography}{9}

\bibitem{ref} \label{ref:BPS} M. Bichler, S. Merting, and A. Uzunoglu, 
“Assigning Course Schedules: About Preference Elicitation, Fairness, and Truthfulness,” 
arXiv preprint arXiv:1812.02630, 2018. [En línea]. Disponible en: 
\url{https://arxiv.org/abs/1812.02630}



\end{thebibliography}

\end{document}